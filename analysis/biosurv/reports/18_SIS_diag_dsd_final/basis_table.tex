% latex table generated in R 3.1.1 by xtable 1.7-4 package
% Mon Dec  8 15:24:01 2014
\begin{longtable}{rrrrrrr}
  \hline
 & 1 & 2 & 3 & 4 & 5 & 6 \\ 
  \hline
PPY & 0.00 & 0.50 & 0.00 & 0.08 & 1.08 & 0.00 \\ 
  KRT6A & 0.14 & 0.00 & 0.12 & 0.00 & 0.00 & 0.47 \\ 
  KRT17 & 0.29 & 0.00 & 0.39 & 0.16 & 0.12 & 0.51 \\ 
  DHRS9 & 0.00 & 0.00 & 1.00 & 0.34 & 0.00 & 0.17 \\ 
  SPP1 & 0.03 & 0.08 & 0.00 & 1.04 & 0.31 & 0.74 \\ 
  ADH1A & 0.07 & 0.44 & 0.01 & 0.10 & 0.66 & 0.00 \\ 
  IGLL3P & 0.17 & 0.15 & 0.00 & 0.00 & 0.76 & 0.00 \\ 
  DKK1 & 0.48 & 0.00 & 0.30 & 0.18 & 0.00 & 0.02 \\ 
  APCS & 0.00 & 0.03 & 0.16 & 0.10 & 0.16 & 0.35 \\ 
  CST6 & 0.07 & 0.00 & 0.20 & 0.00 & 0.07 & 0.63 \\ 
  ANGPTL4 & 0.18 & 0.00 & 0.42 & 0.05 & 0.03 & 0.39 \\ 
  KRT7 & 0.46 & 0.00 & 0.56 & 0.00 & 0.14 & 0.44 \\ 
  PLAU & 0.21 & 0.00 & 0.28 & 0.00 & 0.02 & 0.88 \\ 
  SCGB2A1 & 0.00 & 0.83 & 0.00 & 0.18 & 0.15 & 0.00 \\ 
  CCL19 & 0.00 & 0.00 & 0.00 & 0.00 & 0.95 & 0.00 \\ 
  CYP2S1 & 0.32 & 1.02 & 0.15 & 0.00 & 0.09 & 0.00 \\ 
  SLC2A1 & 0.18 & 0.12 & 1.00 & 0.41 & 0.00 & 0.70 \\ 
  ADM & 0.00 & 0.00 & 0.52 & 0.51 & 0.00 & 0.36 \\ 
  FAM83A & 0.25 & 0.00 & 0.12 & 0.00 & 0.00 & 0.22 \\ 
  FGG & 0.05 & 0.04 & 0.00 & 0.14 & 0.01 & 0.22 \\ 
  KRT6C & 0.12 & 0.00 & 0.00 & 0.00 & 0.00 & 0.16 \\ 
  PHACTR3 & 0.15 & 0.00 & 0.32 & 0.14 & 0.00 & 0.07 \\ 
  C9orf152 & 0.21 & 1.37 & 0.00 & 0.35 & 0.02 & 0.00 \\ 
  ALOX5AP & 0.05 & 0.01 & 0.01 & 1.27 & 0.34 & 0.71 \\ 
  DCBLD2 & 0.40 & 0.00 & 0.12 & 0.00 & 0.14 & 0.84 \\ 
  CIDEC & 0.03 & 0.00 & 0.43 & 0.28 & 0.00 & 0.00 \\ 
  FGB & 0.00 & 0.00 & 0.02 & 0.32 & 0.00 & 0.08 \\ 
  SERPINB3 & 0.00 & 0.00 & 0.18 & 0.18 & 0.00 & 0.05 \\ 
  SLC16A3 & 0.13 & 0.38 & 1.10 & 0.42 & 0.00 & 1.00 \\ 
  FST & 0.00 & 0.00 & 0.16 & 0.00 & 0.04 & 0.49 \\ 
  CAV1 & 0.42 & 0.00 & 0.19 & 0.08 & 0.27 & 0.84 \\ 
  TGFBI & 0.19 & 0.00 & 0.15 & 0.19 & 0.05 & 1.00 \\ 
  COL12A1 & 0.00 & 0.13 & 0.03 & 0.53 & 0.19 & 1.65 \\ 
  SLC2A3 & 0.00 & 0.00 & 0.34 & 0.76 & 0.33 & 0.72 \\ 
  SUGCT & 0.00 & 0.03 & 0.00 & 0.63 & 0.13 & 0.93 \\ 
  IL1R2 & 0.04 & 0.25 & 0.43 & 0.23 & 0.00 & 0.06 \\ 
  TCEA3 & 0.00 & 0.89 & 0.26 & 0.09 & 0.62 & 0.00 \\ 
  RAP1GAP & 0.00 & 1.01 & 0.47 & 0.28 & 0.75 & 0.00 \\ 
  PXDN & 0.00 & 0.00 & 0.38 & 0.59 & 0.31 & 1.19 \\ 
  FRZB & 0.09 & 0.24 & 0.00 & 0.54 & 1.50 & 0.00 \\ 
  IL20RB & 0.26 & 0.00 & 0.31 & 0.00 & 0.00 & 0.68 \\ 
  PLEKHS1 & 0.00 & 0.64 & 0.34 & 0.09 & 0.28 & 0.02 \\ 
  HSPB6 & 0.00 & 0.15 & 0.13 & 0.00 & 1.31 & 0.31 \\ 
  KANK4 & 0.00 & 0.00 & 0.20 & 0.47 & 0.00 & 1.23 \\ 
  COL7A1 & 0.00 & 0.00 & 0.59 & 0.00 & 0.00 & 0.59 \\ 
  C5orf46 & 0.00 & 0.00 & 0.00 & 1.06 & 0.13 & 1.04 \\ 
  VSTM2L & 0.32 & 0.00 & 0.94 & 0.00 & 0.05 & 0.07 \\ 
  PTGES & 0.57 & 0.02 & 0.57 & 0.07 & 0.00 & 0.56 \\ 
  FSCN1 & 0.37 & 0.07 & 1.06 & 0.13 & 0.14 & 0.74 \\ 
  CTSV & 0.30 & 0.04 & 0.26 & 0.02 & 0.02 & 0.18 \\ 
  SPOCK1 & 0.12 & 0.00 & 0.03 & 0.52 & 0.34 & 1.27 \\ 
  RGS5 & 0.00 & 0.43 & 0.05 & 0.08 & 0.58 & 0.09 \\ 
  PHLDA1 & 0.08 & 0.14 & 0.72 & 0.13 & 0.62 & 1.50 \\ 
  IGFBP1 & 0.27 & 0.00 & 0.23 & 0.03 & 0.00 & 0.01 \\ 
  BAMBI & 0.11 & 0.00 & 0.84 & 0.39 & 0.24 & 0.17 \\ 
  FLRT3 & 0.79 & 0.13 & 0.51 & 0.28 & 0.22 & 0.31 \\ 
  DSG3 & 0.18 & 0.00 & 0.21 & 0.00 & 0.00 & 0.54 \\ 
  ANGPTL2 & 0.00 & 0.00 & 0.37 & 0.87 & 0.18 & 0.92 \\ 
  ST6GAL1 & 0.17 & 0.84 & 0.00 & 0.23 & 0.67 & 0.09 \\ 
  SLC40A1 & 0.00 & 0.89 & 0.00 & 0.58 & 0.24 & 0.16 \\ 
  EMP3 & 0.25 & 0.00 & 0.46 & 0.16 & 0.22 & 0.56 \\ 
  RAB31 & 0.11 & 0.00 & 0.26 & 0.87 & 0.76 & 1.19 \\ 
  ST6GALNAC1 & 0.04 & 1.00 & 0.08 & 0.12 & 0.00 & 0.10 \\ 
  ACKR3 & 0.00 & 0.00 & 0.38 & 0.36 & 0.21 & 0.58 \\ 
  SLC12A2 & 0.04 & 0.91 & 0.34 & 0.10 & 0.49 & 0.18 \\ 
  ANKRD22 & 0.41 & 1.35 & 0.17 & 0.27 & 0.04 & 0.22 \\ 
  ENO2 & 0.36 & 0.34 & 0.79 & 0.03 & 0.00 & 0.94 \\ 
  EPHX2 & 0.00 & 0.59 & 0.11 & 0.17 & 0.68 & 0.00 \\ 
  MCEMP1 & 0.00 & 0.00 & 0.00 & 0.61 & 0.00 & 0.30 \\ 
  CDA & 0.29 & 0.00 & 0.34 & 0.00 & 0.00 & 0.70 \\ 
  PLIN2 & 0.31 & 0.00 & 0.08 & 1.02 & 0.47 & 0.21 \\ 
  SERPINH1 & 0.00 & 0.01 & 0.39 & 0.22 & 0.43 & 1.02 \\ 
  FAM134B & 0.00 & 0.82 & 0.00 & 0.23 & 0.21 & 0.00 \\ 
  NFIX & 0.00 & 0.88 & 0.14 & 0.00 & 1.39 & 0.80 \\ 
  LYNX1 & 0.03 & 0.00 & 0.26 & 0.17 & 0.00 & 0.10 \\ 
  LDHA & 0.65 & 0.47 & 0.00 & 0.32 & 0.05 & 1.17 \\ 
  SOD2 & 0.58 & 0.12 & 0.00 & 0.47 & 0.40 & 0.17 \\ 
  PCDH20 & 0.00 & 0.43 & 0.00 & 0.15 & 0.00 & 0.00 \\ 
  ITGA5 & 0.00 & 0.00 & 0.48 & 0.27 & 0.12 & 0.68 \\ 
  ZNF185 & 0.25 & 0.17 & 1.02 & 0.48 & 0.00 & 0.72 \\ 
  PLOD2 & 0.15 & 0.09 & 0.24 & 0.29 & 0.17 & 0.89 \\ 
  TNFRSF6B & 0.63 & 0.00 & 0.07 & 0.18 & 0.00 & 0.39 \\ 
  MME & 0.00 & 0.00 & 0.06 & 0.45 & 0.04 & 0.58 \\ 
  MRAP2 & 0.04 & 0.78 & 0.00 & 0.22 & 0.23 & 0.00 \\ 
  PLAC9 & 0.07 & 0.00 & 0.00 & 0.11 & 1.29 & 0.08 \\ 
  ERRFI1 & 0.16 & 0.03 & 0.55 & 0.35 & 0.29 & 0.79 \\ 
  PP7080 & 0.10 & 0.97 & 0.00 & 0.04 & 0.00 & 0.00 \\ 
  DSG2 & 0.43 & 0.57 & 0.18 & 0.51 & 0.04 & 0.71 \\ 
  APCDD1 & 0.00 & 0.14 & 0.15 & 0.13 & 0.60 & 0.84 \\ 
  PRKCDBP & 0.26 & 0.00 & 1.02 & 0.51 & 0.26 & 0.59 \\ 
  SULF2 & 0.17 & 0.15 & 0.46 & 0.19 & 0.39 & 0.77 \\ 
  TUBA1C & 1.31 & 0.55 & 0.54 & 0.53 & 0.27 & 0.50 \\ 
  PCOLCE2 & 0.00 & 0.01 & 0.12 & 0.54 & 0.00 & 0.05 \\ 
  LAMA5 & 0.37 & 0.08 & 1.02 & 0.00 & 0.34 & 0.18 \\ 
  P4HA1 & 0.04 & 0.10 & 0.41 & 0.84 & 0.00 & 0.55 \\ 
  RASL11B & 0.00 & 0.19 & 0.07 & 0.22 & 1.21 & 0.31 \\ 
  KYNU & 0.61 & 0.09 & 0.07 & 0.54 & 0.00 & 0.28 \\ 
  CTSL & 0.39 & 0.00 & 0.20 & 1.18 & 0.47 & 0.22 \\ 
  MARCKSL1 & 0.15 & 1.34 & 0.30 & 0.00 & 0.00 & 0.26 \\ 
  PRC1 & 0.96 & 0.35 & 0.04 & 0.04 & 0.00 & 0.32 \\ 
  C1QTNF6 & 0.00 & 0.00 & 0.59 & 0.62 & 0.22 & 0.97 \\ 
  CCR7 & 0.06 & 0.00 & 0.00 & 0.00 & 1.05 & 0.00 \\ 
  HRASLS2 & 0.33 & 0.00 & 0.30 & 0.22 & 0.00 & 0.00 \\ 
  CHN2 & 0.00 & 0.50 & 0.00 & 0.34 & 0.44 & 0.00 \\ 
  PYGL & 0.08 & 0.00 & 0.31 & 0.34 & 0.14 & 0.74 \\ 
  MELK & 1.02 & 0.29 & 0.00 & 0.23 & 0.01 & 0.22 \\ 
  LOX & 0.21 & 0.00 & 0.08 & 0.39 & 0.09 & 0.92 \\ 
  CDC45 & 0.96 & 0.08 & 0.11 & 0.34 & 0.03 & 0.00 \\ 
  AXIN2 & 0.00 & 0.52 & 0.44 & 0.13 & 0.81 & 0.29 \\ 
  ATL3 & 0.64 & 0.03 & 0.16 & 0.49 & 0.25 & 0.29 \\ 
  CAMK1G & 0.09 & 0.24 & 0.00 & 0.03 & 0.88 & 0.00 \\ 
  ABLIM1 & 0.01 & 0.91 & 0.32 & 0.00 & 0.61 & 0.34 \\ 
  TRIM2 & 0.13 & 1.15 & 0.31 & 0.31 & 0.36 & 0.00 \\ 
  TWIST1 & 0.00 & 0.00 & 0.20 & 0.91 & 0.12 & 1.20 \\ 
  ARSD & 0.15 & 1.24 & 0.19 & 0.00 & 0.22 & 0.14 \\ 
  CEBPB & 0.07 & 0.07 & 1.29 & 0.53 & 0.51 & 0.81 \\ 
  CEP55 & 1.42 & 0.33 & 0.00 & 0.17 & 0.00 & 0.46 \\ 
  GINS2 & 1.08 & 0.18 & 0.39 & 0.07 & 0.00 & 0.00 \\ 
  MCM4 & 1.28 & 0.14 & 0.31 & 0.03 & 0.01 & 0.13 \\ 
  PPP1R3C & 0.00 & 0.02 & 0.13 & 0.37 & 0.03 & 0.26 \\ 
  MTRNR2L1 & 0.28 & 0.56 & 0.49 & 0.07 & 0.55 & 0.00 \\ 
  CDK12 & 0.19 & 0.28 & 0.00 & 0.08 & 0.83 & 0.00 \\ 
  BIRC5 & 1.38 & 0.17 & 0.37 & 0.55 & 0.00 & 0.24 \\ 
  SPHK1 & 0.26 & 0.00 & 0.27 & 0.09 & 0.62 & 1.41 \\ 
  A4GALT & 0.03 & 0.00 & 1.30 & 0.08 & 0.36 & 0.52 \\ 
  ICAM2 & 0.50 & 0.20 & 0.48 & 0.31 & 0.40 & 0.13 \\ 
  ANKRD37 & 0.06 & 0.18 & 0.22 & 0.72 & 0.01 & 0.57 \\ 
  STK39 & 0.15 & 1.00 & 0.24 & 0.14 & 0.08 & 0.12 \\ 
  ASPM & 1.17 & 0.39 & 0.20 & 0.17 & 0.04 & 0.04 \\ 
  PFKFB4 & 0.55 & 0.22 & 0.68 & 0.43 & 0.14 & 0.29 \\ 
  IDH2 & 0.71 & 0.43 & 0.40 & 0.21 & 0.33 & 0.23 \\ 
  SGSM1 & 0.00 & 0.93 & 0.08 & 0.02 & 0.84 & 0.00 \\ 
  SELENBP1 & 0.00 & 1.20 & 0.36 & 0.20 & 0.26 & 0.00 \\ 
  P4HA2 & 0.32 & 0.17 & 0.12 & 0.54 & 0.11 & 0.74 \\ 
  LMO3 & 0.00 & 0.11 & 0.00 & 0.01 & 1.18 & 0.01 \\ 
  KLHL5 & 0.42 & 0.16 & 0.00 & 0.35 & 0.70 & 1.14 \\ 
  HIPK2 & 0.26 & 1.25 & 0.07 & 0.24 & 0.52 & 0.00 \\ 
  NAMPT & 0.34 & 0.00 & 0.05 & 0.75 & 0.32 & 0.35 \\ 
  NCAPG & 1.61 & 0.44 & 0.00 & 0.00 & 0.00 & 0.52 \\ 
  PLOD1 & 0.06 & 0.00 & 1.21 & 0.75 & 0.37 & 0.80 \\ 
  C2orf70 & 0.11 & 1.09 & 0.02 & 0.00 & 0.00 & 0.00 \\ 
  RERGL & 0.24 & 0.00 & 0.00 & 0.11 & 1.18 & 0.00 \\ 
  CFDP1 & 0.35 & 0.55 & 0.74 & 0.67 & 0.00 & 0.26 \\ 
  RACGAP1 & 1.37 & 0.37 & 0.14 & 0.19 & 0.07 & 0.33 \\ 
  SNRPB & 0.99 & 0.08 & 0.41 & 0.90 & 0.02 & 0.00 \\ 
  CLEC3B & 0.06 & 0.07 & 0.12 & 0.01 & 0.81 & 0.00 \\ 
  ANLN & 1.17 & 0.24 & 0.08 & 0.08 & 0.00 & 0.72 \\ 
  ZFPM1 & 0.00 & 1.22 & 0.29 & 0.00 & 0.43 & 0.15 \\ 
  UPP1 & 0.55 & 0.00 & 0.79 & 0.43 & 0.16 & 0.11 \\ 
  AURKB & 1.00 & 0.11 & 0.14 & 0.00 & 0.01 & 0.00 \\ 
  SYNE2 & 0.00 & 0.88 & 0.24 & 0.00 & 0.28 & 0.28 \\ 
  SOBP & 0.00 & 0.20 & 0.81 & 0.10 & 1.36 & 0.00 \\ 
  GAPDH & 0.48 & 0.39 & 0.83 & 0.24 & 0.00 & 0.72 \\ 
  SERTAD2 & 0.29 & 0.14 & 0.90 & 0.99 & 0.49 & 0.44 \\ 
  TPX2 & 1.32 & 0.15 & 0.04 & 0.15 & 0.04 & 0.11 \\ 
  POC1A & 1.38 & 0.33 & 0.32 & 0.47 & 0.00 & 0.00 \\ 
  PDLIM7 & 0.20 & 0.00 & 0.41 & 0.37 & 0.11 & 0.68 \\ 
  TSTD1 & 0.17 & 1.22 & 0.48 & 0.07 & 0.45 & 0.02 \\ 
  PLIN3 & 0.34 & 0.26 & 0.97 & 0.93 & 0.14 & 0.41 \\ 
  IL33 & 0.24 & 0.04 & 0.00 & 0.13 & 0.68 & 0.00 \\ 
  CA8 & 0.00 & 0.69 & 0.05 & 0.01 & 0.05 & 0.00 \\ 
  SAMD5 & 0.13 & 0.54 & 0.00 & 0.00 & 0.09 & 0.00 \\ 
  NFIA & 0.12 & 0.84 & 0.00 & 0.39 & 1.50 & 0.27 \\ 
  KCTD5 & 0.38 & 0.51 & 1.13 & 0.61 & 0.00 & 0.00 \\ 
  CCNB1 & 1.43 & 0.46 & 0.13 & 0.25 & 0.02 & 0.36 \\ 
  TM9SF3 & 0.00 & 1.08 & 0.22 & 0.00 & 0.16 & 0.21 \\ 
  KIF20A & 1.37 & 0.29 & 0.21 & 0.23 & 0.00 & 0.29 \\ 
  PROSER2 & 0.93 & 0.18 & 0.40 & 0.37 & 0.27 & 0.40 \\ 
  COLGALT1 & 0.40 & 0.16 & 0.62 & 0.43 & 0.16 & 0.88 \\ 
  PPM1H & 0.00 & 0.85 & 0.46 & 0.27 & 0.24 & 0.00 \\ 
  NCAPD2 & 1.38 & 0.41 & 0.16 & 0.12 & 0.20 & 0.32 \\ 
  PREP & 0.06 & 0.98 & 0.30 & 0.20 & 0.02 & 0.00 \\ 
  DPY19L1 & 0.34 & 0.36 & 0.30 & 0.54 & 0.08 & 0.51 \\ 
  CKAP2L & 1.78 & 0.22 & 0.27 & 0.03 & 0.00 & 0.09 \\ 
  ZBED2 & 0.16 & 0.00 & 0.18 & 0.00 & 0.00 & 0.64 \\ 
  MIR99AHG & 0.04 & 0.28 & 0.39 & 0.45 & 1.79 & 0.22 \\ 
  P2RY2 & 0.18 & 0.03 & 0.77 & 0.22 & 0.00 & 0.50 \\ 
  KIF2C & 0.80 & 0.13 & 0.11 & 0.01 & 0.00 & 0.00 \\ 
  PPP1R14B & 0.37 & 0.26 & 0.78 & 0.00 & 0.37 & 0.59 \\ 
  GPC3 & 0.10 & 0.23 & 0.00 & 0.00 & 1.27 & 0.00 \\ 
  MAP3K8 & 0.20 & 0.00 & 0.07 & 0.31 & 0.56 & 0.43 \\ 
  NMB & 0.21 & 0.19 & 0.66 & 0.79 & 0.00 & 0.36 \\ 
  RAVER2 & 0.20 & 0.91 & 0.05 & 0.09 & 0.27 & 0.06 \\ 
  SPIN4 & 0.85 & 0.32 & 0.80 & 0.39 & 0.22 & 0.40 \\ 
  AMOT & 0.07 & 0.82 & 0.14 & 0.52 & 0.43 & 0.57 \\ 
  POP5 & 0.56 & 0.51 & 1.52 & 0.23 & 0.11 & 0.18 \\ 
  COLGALT2 & 0.00 & 0.60 & 0.00 & 0.02 & 0.00 & 0.00 \\ 
  DCUN1D5 & 1.36 & 0.08 & 0.00 & 0.86 & 0.96 & 0.72 \\ 
  DNAJC9 & 0.78 & 0.11 & 0.37 & 0.12 & 0.13 & 0.15 \\ 
  KCTD10 & 0.38 & 0.13 & 0.29 & 0.44 & 0.51 & 0.79 \\ 
  MIF & 0.43 & 0.33 & 0.96 & 0.44 & 0.00 & 0.68 \\ 
  SLAMF9 & 0.04 & 0.00 & 0.00 & 0.67 & 0.00 & 0.07 \\ 
  MCOLN2 & 0.20 & 0.28 & 0.00 & 0.00 & 0.94 & 0.00 \\ 
  CSNK1D & 0.21 & 0.38 & 1.56 & 0.48 & 0.16 & 0.23 \\ 
  TMED1 & 0.26 & 0.34 & 1.15 & 0.83 & 0.49 & 0.28 \\ 
  CADPS2 & 0.26 & 1.29 & 0.00 & 0.55 & 1.02 & 0.57 \\ 
  MEOX1 & 0.00 & 0.05 & 0.16 & 0.04 & 0.96 & 0.00 \\ 
  GIMAP2 & 0.15 & 0.72 & 0.00 & 0.66 & 0.77 & 0.00 \\ 
  RFC5 & 1.08 & 0.24 & 0.00 & 0.52 & 0.16 & 0.31 \\ 
  CARHSP1 & 0.75 & 0.53 & 0.87 & 0.90 & 0.26 & 0.00 \\ 
  SLC15A1 & 0.00 & 0.00 & 0.48 & 0.00 & 0.06 & 0.06 \\ 
  BCL11B & 0.20 & 0.92 & 0.23 & 0.24 & 0.42 & 0.00 \\ 
  CDK2 & 1.06 & 0.25 & 0.01 & 0.52 & 0.33 & 0.33 \\ 
  KIAA1549L & 0.38 & 0.08 & 0.26 & 0.66 & 0.15 & 0.64 \\ 
  HJURP & 1.33 & 0.24 & 0.23 & 0.02 & 0.00 & 0.00 \\ 
  FYN & 0.01 & 0.52 & 0.12 & 0.13 & 1.69 & 0.87 \\ 
  RNF103 & 0.03 & 1.25 & 0.17 & 0.55 & 0.29 & 0.06 \\ 
  ACYP2 & 0.25 & 0.89 & 0.00 & 0.23 & 0.85 & 0.41 \\ 
  CD70 & 0.09 & 0.00 & 0.21 & 0.36 & 0.00 & 0.43 \\ 
  PPAPDC1A & 0.00 & 0.00 & 0.00 & 0.76 & 0.00 & 1.22 \\ 
  TPD52L2 & 0.63 & 0.16 & 1.31 & 0.65 & 0.44 & 0.23 \\ 
  TOM1 & 0.00 & 0.10 & 1.49 & 0.81 & 0.68 & 0.52 \\ 
  DERA & 1.18 & 0.20 & 0.46 & 0.60 & 0.29 & 0.32 \\ 
  TREM1 & 0.05 & 0.00 & 0.09 & 0.71 & 0.00 & 0.30 \\ 
  UFC1 & 0.00 & 1.19 & 0.25 & 0.47 & 0.30 & 0.00 \\ 
  TCTA & 0.00 & 0.75 & 0.82 & 0.09 & 0.98 & 0.02 \\ 
  ALDH5A1 & 0.10 & 0.99 & 0.55 & 0.06 & 0.90 & 0.22 \\ 
  KNTC1 & 1.07 & 0.14 & 0.44 & 0.08 & 0.15 & 0.28 \\ 
  XXYLT1 & 0.24 & 0.00 & 1.05 & 1.08 & 0.46 & 0.87 \\ 
  SMOX & 0.37 & 0.29 & 1.43 & 1.00 & 0.18 & 0.00 \\ 
  ARFGAP3 & 0.03 & 0.30 & 0.54 & 0.84 & 0.49 & 0.54 \\ 
  SEPW1 & 0.03 & 0.95 & 1.24 & 0.00 & 0.63 & 0.56 \\ 
  ANKLE2 & 0.75 & 0.14 & 0.62 & 0.51 & 0.19 & 0.38 \\ 
  TLE4 & 0.05 & 0.88 & 0.07 & 0.33 & 0.90 & 0.47 \\ 
  RBMS2 & 0.61 & 0.15 & 0.00 & 0.40 & 0.32 & 0.89 \\ 
  AKR1A1 & 0.25 & 1.08 & 0.26 & 0.29 & 0.66 & 0.45 \\ 
  RERE & 0.05 & 0.74 & 0.62 & 0.00 & 0.99 & 0.42 \\ 
  ATAD2 & 0.94 & 0.07 & 0.11 & 0.03 & 0.11 & 0.31 \\ 
  SPOCD1 & 0.00 & 0.00 & 0.18 & 0.21 & 0.00 & 0.76 \\ 
  DYNC2H1 & 0.00 & 1.61 & 0.15 & 0.00 & 0.76 & 0.67 \\ 
  CAPN6 & 0.00 & 0.75 & 0.00 & 0.23 & 0.64 & 0.00 \\ 
  RPIA & 0.46 & 1.35 & 0.22 & 0.19 & 0.46 & 0.00 \\ 
  P2RY8 & 0.23 & 0.07 & 0.00 & 0.28 & 1.66 & 0.00 \\ 
  ARHGEF19 & 0.08 & 0.08 & 1.20 & 0.52 & 0.45 & 0.51 \\ 
  ARL4C & 0.00 & 0.02 & 0.30 & 0.49 & 0.30 & 1.23 \\ 
  CHAF1B & 0.99 & 0.30 & 0.20 & 0.02 & 0.52 & 0.10 \\ 
  FHDC1 & 0.18 & 1.24 & 0.22 & 0.02 & 0.00 & 0.05 \\ 
  POLA2 & 0.84 & 0.22 & 0.33 & 0.13 & 0.21 & 0.00 \\ 
  AGRP & 0.00 & 0.00 & 0.00 & 0.68 & 0.00 & 0.17 \\ 
  RPA2 & 0.47 & 0.70 & 0.70 & 0.41 & 1.42 & 0.24 \\ 
  MRPL24 & 0.16 & 1.13 & 0.22 & 0.12 & 0.22 & 0.18 \\ 
  PRDM16 & 0.00 & 1.12 & 0.00 & 0.00 & 0.53 & 0.09 \\ 
  POU2AF1 & 0.06 & 0.47 & 0.00 & 0.00 & 0.92 & 0.00 \\ 
  MC1R & 0.10 & 0.13 & 1.08 & 0.87 & 0.47 & 0.13 \\ 
  TNFRSF17 & 0.03 & 0.05 & 0.00 & 0.08 & 0.58 & 0.00 \\ 
  FAH & 0.68 & 0.42 & 0.36 & 0.21 & 0.32 & 0.39 \\ 
  HSP90B1 & 0.53 & 0.46 & 0.78 & 0.90 & 0.30 & 0.38 \\ 
  TRAPPC2 & 0.51 & 1.08 & 0.00 & 0.49 & 0.62 & 0.14 \\ 
  ARHGAP24 & 0.06 & 1.06 & 0.02 & 0.75 & 1.10 & 0.62 \\ 
  ABHD16A & 0.66 & 0.72 & 0.00 & 0.00 & 0.52 & 0.22 \\ 
  TMTC4 & 0.00 & 1.29 & 0.33 & 0.21 & 0.20 & 0.28 \\ 
  SCYL2 & 0.70 & 0.39 & 0.00 & 0.98 & 0.41 & 0.96 \\ 
  TOR2A & 0.00 & 0.99 & 0.48 & 0.20 & 0.53 & 0.00 \\ 
  IKBIP & 0.29 & 0.00 & 0.30 & 1.12 & 0.15 & 0.47 \\ 
  DENND1A & 0.82 & 0.00 & 0.25 & 0.19 & 0.00 & 0.18 \\ 
  BCKDK & 0.22 & 0.29 & 0.87 & 1.07 & 0.40 & 0.11 \\ 
  KIAA0513 & 0.08 & 1.04 & 0.17 & 0.32 & 0.59 & 0.00 \\ 
  CNNM1 & 0.00 & 0.87 & 0.41 & 0.00 & 0.09 & 0.00 \\ 
  VPS35 & 0.39 & 1.39 & 0.00 & 0.53 & 0.00 & 0.25 \\ 
  ZPLD1 & 0.00 & 0.00 & 0.19 & 0.03 & 0.03 & 0.11 \\ 
  CHEK1 & 1.52 & 0.16 & 0.00 & 0.00 & 0.11 & 0.27 \\ 
  PEX11B & 0.11 & 1.35 & 0.00 & 0.53 & 0.29 & 0.25 \\ 
  BTN3A1 & 0.66 & 0.71 & 0.07 & 0.25 & 0.99 & 0.30 \\ 
  FBXO22 & 0.50 & 0.36 & 0.00 & 0.58 & 0.00 & 0.31 \\ 
  BBS2 & 0.25 & 1.14 & 0.00 & 0.22 & 1.00 & 1.16 \\ 
  DCAF8 & 0.00 & 1.14 & 0.48 & 0.11 & 0.53 & 0.19 \\ 
  ITPKB & 0.00 & 0.83 & 0.61 & 0.00 & 1.19 & 0.67 \\ 
  SH3GL1 & 0.12 & 0.11 & 1.01 & 1.25 & 0.22 & 0.00 \\ 
  PBXIP1 & 0.00 & 0.51 & 0.41 & 0.00 & 0.44 & 0.17 \\ 
  GAB2 & 0.04 & 0.74 & 0.38 & 0.64 & 1.36 & 0.27 \\ 
  NACC2 & 0.53 & 0.00 & 0.72 & 0.25 & 0.00 & 0.11 \\ 
  EXOSC8 & 0.93 & 0.60 & 0.28 & 1.02 & 0.37 & 0.15 \\ 
  ATF7IP2 & 0.00 & 0.20 & 0.12 & 0.00 & 0.03 & 0.00 \\ 
  MCM10 & 1.14 & 0.14 & 0.00 & 0.01 & 0.00 & 0.08 \\ 
  PGAM5 & 0.92 & 0.00 & 0.39 & 0.49 & 0.00 & 0.00 \\ 
  AKIP1 & 0.64 & 0.24 & 0.60 & 0.71 & 0.78 & 0.72 \\ 
  STAT5B & 0.00 & 0.91 & 0.32 & 0.06 & 1.31 & 0.22 \\ 
  KIF14 & 1.12 & 0.36 & 0.20 & 0.43 & 0.00 & 0.13 \\ 
  FAM189A2 & 0.00 & 1.00 & 0.00 & 0.02 & 0.11 & 0.00 \\ 
  GNPAT & 0.17 & 0.95 & 0.14 & 0.44 & 0.18 & 0.19 \\ 
  PAX8 & 0.77 & 0.00 & 0.56 & 0.00 & 0.00 & 0.00 \\ 
  GABPB1 & 0.74 & 0.20 & 0.00 & 0.74 & 0.22 & 0.67 \\ 
  TARBP2 & 0.68 & 0.38 & 1.22 & 0.61 & 0.18 & 0.00 \\ 
  ABHD5 & 0.15 & 0.75 & 0.00 & 0.75 & 0.40 & 1.17 \\ 
  NUP155 & 1.13 & 0.41 & 0.06 & 0.33 & 0.23 & 0.46 \\ 
  FAM120AOS & 0.18 & 1.05 & 0.00 & 0.28 & 0.71 & 0.57 \\ 
  CATSPER1 & 0.12 & 0.00 & 0.92 & 0.00 & 0.00 & 0.10 \\ 
  RFK & 0.00 & 0.66 & 0.12 & 0.00 & 0.43 & 0.21 \\ 
  CIDECP & 0.11 & 0.02 & 0.52 & 0.28 & 0.11 & 0.00 \\ 
  CACHD1 & 0.00 & 0.69 & 0.02 & 0.00 & 1.08 & 0.49 \\ 
  NR0B2 & 0.00 & 0.84 & 0.00 & 0.00 & 0.14 & 0.00 \\ 
  TMEM26 & 0.04 & 0.02 & 0.10 & 0.49 & 0.22 & 1.45 \\ 
  NELFE & 0.94 & 0.23 & 0.59 & 0.86 & 0.36 & 0.08 \\ 
  ZSCAN16 & 0.30 & 1.45 & 0.00 & 0.02 & 0.51 & 0.51 \\ 
  FAM91A1 & 0.98 & 0.20 & 0.16 & 0.79 & 0.00 & 0.27 \\ 
  PHOSPHO2 & 0.34 & 1.07 & 0.00 & 0.47 & 0.41 & 0.05 \\ 
  KCNQ3 & 0.00 & 0.13 & 0.17 & 0.78 & 0.09 & 0.52 \\ 
  RHOF & 0.75 & 0.17 & 0.48 & 0.14 & 0.00 & 0.59 \\ 
  COX4I2 & 0.00 & 0.17 & 0.07 & 0.00 & 0.99 & 0.33 \\ 
  MARS2 & 0.75 & 1.02 & 0.00 & 0.40 & 0.50 & 0.00 \\ 
  BOC & 0.00 & 0.00 & 0.32 & 0.00 & 1.61 & 0.00 \\ 
  ZSCAN32 & 0.35 & 1.16 & 0.50 & 0.30 & 0.73 & 0.24 \\ 
  PCF11 & 0.26 & 0.94 & 0.25 & 0.10 & 1.11 & 0.41 \\ 
  SEC23IP & 0.34 & 1.30 & 0.00 & 0.53 & 0.36 & 0.46 \\ 
  E2F7 & 1.04 & 0.00 & 0.03 & 0.02 & 0.00 & 0.54 \\ 
  COL5A3 & 0.00 & 0.00 & 0.18 & 0.04 & 0.07 & 1.03 \\ 
  SNORA11D & 0.08 & 0.27 & 0.48 & 0.44 & 0.00 & 0.27 \\ 
  OAZ1 & 0.86 & 0.59 & 0.66 & 1.12 & 0.52 & 0.59 \\ 
  LETM2 & 0.44 & 0.00 & 0.39 & 0.00 & 0.00 & 0.28 \\ 
  EIF2AK3 & 0.18 & 1.27 & 0.00 & 0.38 & 0.61 & 0.33 \\ 
  ST3GAL2 & 0.34 & 0.00 & 0.80 & 1.07 & 0.44 & 0.00 \\ 
  PRR11 & 0.82 & 0.05 & 0.23 & 0.00 & 0.00 & 0.09 \\ 
  ELMOD3 & 0.00 & 1.16 & 0.69 & 0.39 & 0.53 & 0.09 \\ 
  CNIH3 & 0.00 & 0.06 & 0.00 & 0.32 & 0.00 & 0.60 \\ 
  B3GALTL & 0.36 & 0.33 & 0.56 & 0.38 & 0.49 & 0.77 \\ 
  PRMT7 & 0.14 & 1.50 & 0.44 & 0.00 & 0.18 & 0.22 \\ 
  FGD6 & 0.55 & 0.00 & 0.13 & 0.14 & 0.00 & 0.50 \\ 
  TRERF1 & 0.49 & 0.29 & 0.38 & 0.13 & 0.05 & 0.13 \\ 
  RALGAPB & 1.00 & 0.50 & 0.29 & 0.76 & 0.26 & 0.80 \\ 
  UHRF2 & 0.15 & 0.29 & 0.33 & 0.50 & 0.66 & 1.10 \\ 
  GOLM1 & 0.00 & 0.71 & 0.12 & 0.05 & 0.00 & 0.00 \\ 
  PAX8-AS1 & 0.57 & 0.04 & 0.34 & 0.07 & 0.01 & 0.00 \\ 
  THSD7B & 0.09 & 0.20 & 0.00 & 0.29 & 0.96 & 0.11 \\ 
  TAF5L & 0.22 & 1.06 & 0.18 & 0.24 & 0.23 & 0.22 \\ 
  PPP1R12B & 0.17 & 0.32 & 0.78 & 0.63 & 0.03 & 0.49 \\ 
  LINC01184 & 0.63 & 0.80 & 0.00 & 0.34 & 0.81 & 0.00 \\ 
  RFX2 & 0.00 & 0.22 & 0.24 & 0.00 & 0.46 & 0.30 \\ 
  WNT2B & 0.09 & 0.11 & 0.00 & 0.01 & 0.45 & 0.00 \\ 
  TOM1L2 & 0.19 & 0.00 & 0.63 & 0.33 & 0.05 & 0.23 \\ 
  TNFRSF10D & 0.15 & 0.11 & 0.66 & 0.46 & 0.00 & 0.18 \\ 
  GATA6 & 0.05 & 0.88 & 0.09 & 0.14 & 0.19 & 0.00 \\ 
  SDIM1 & 0.00 & 0.05 & 0.24 & 0.00 & 0.50 & 0.00 \\ 
  ZNF658 & 0.00 & 0.88 & 0.00 & 0.00 & 0.91 & 0.28 \\ 
  IFT140 & 0.00 & 1.09 & 0.52 & 0.00 & 0.26 & 0.07 \\ 
  LGALS9B & 0.11 & 1.02 & 0.00 & 0.00 & 0.35 & 0.49 \\ 
  LMTK2 & 0.74 & 0.36 & 0.31 & 0.53 & 0.02 & 0.24 \\ 
  FER & 0.50 & 0.10 & 0.18 & 0.44 & 0.18 & 0.87 \\ 
  NRP2 & 0.15 & 0.00 & 0.50 & 0.00 & 0.00 & 0.05 \\ 
  EYA3 & 0.00 & 0.09 & 0.53 & 0.00 & 0.00 & 0.91 \\ 
  ZNF565 & 0.07 & 0.29 & 0.07 & 0.06 & 0.24 & 0.08 \\ 
  GATC & 1.02 & 0.11 & 0.00 & 0.48 & 0.07 & 0.47 \\ 
  CCDC88A & 0.00 & 0.17 & 0.47 & 0.01 & 0.80 & 1.02 \\ 
  USP30 & 0.54 & 0.14 & 0.39 & 0.00 & 0.08 & 0.00 \\ 
  LOC100506562 & 0.58 & 0.29 & 0.60 & 0.60 & 0.11 & 0.11 \\ 
  RMND5A & 0.27 & 0.12 & 0.26 & 0.71 & 0.00 & 0.07 \\ 
  FBXW8 & 0.25 & 0.26 & 0.66 & 0.93 & 0.18 & 0.33 \\ 
  EDIL3 & 0.00 & 0.00 & 0.00 & 0.86 & 0.01 & 0.82 \\ 
  A4GNT & 0.00 & 0.74 & 0.05 & 0.05 & 0.37 & 0.07 \\ 
  ORC1 & 0.98 & 0.32 & 0.16 & 0.95 & 0.12 & 0.01 \\ 
  FEM1B & 0.30 & 0.30 & 0.00 & 0.00 & 0.08 & 1.42 \\ 
  SLC30A3 & 0.45 & 0.50 & 0.08 & 0.21 & 0.66 & 0.07 \\ 
  C1orf56 & 0.00 & 0.87 & 0.00 & 0.37 & 0.11 & 0.36 \\ 
  NEURL2 & 0.69 & 0.12 & 0.00 & 0.26 & 0.72 & 0.43 \\ 
  PGBD3 & 0.62 & 0.36 & 0.43 & 0.20 & 0.56 & 0.74 \\ 
  PTPN21 & 0.27 & 0.17 & 0.32 & 0.49 & 0.27 & 0.84 \\ 
  LCNL1 & 0.11 & 0.28 & 0.01 & 0.27 & 0.53 & 0.00 \\ 
  ACE & 0.03 & 0.83 & 0.05 & 0.00 & 0.00 & 0.18 \\ 
  NPM1 & 0.00 & 1.05 & 0.00 & 0.00 & 0.08 & 0.04 \\ 
  RGS3 & 0.24 & 0.12 & 0.00 & 0.81 & 0.23 & 0.32 \\ 
  PIGL & 1.06 & 0.15 & 0.56 & 0.30 & 0.24 & 0.00 \\ 
  GPR176 & 0.43 & 0.31 & 0.00 & 0.74 & 0.37 & 0.59 \\ 
   \hline
\hline
\end{longtable}
