\documentclass[thesis.tex]{subfiles}
\begin{document}
\chapter{Survival Processes in Pancreatic Cancer}
\label{ch:prognostic}

Outline ideas:
\begin{itemize}
  \item Overall thesis for this work: That specific molecular processes control survival following PC resection, and that these processes are not apparent from examining CPVs, but can be identified and detected using GEX data.
  \item Introduction
  \begin{itemize}
    \item General background on outcome with PC, with focus on post-op.  Make the note of the wide range of survival times.  Segue into reasons for this variability...
    \item Background on predictive CPVs in PC.  MSKCC.
    \item What is known about molecular risk factors so far?
    \begin{itemize}
      \item There are some very specific cases, eg HER2
      \item On more broad sigs, there are some (eg Collisson), but:
      \begin{itemize}
        \item Generated backwards, generally by unsupervised clustering
        \item Cluster-based sigs don't appropriately capture smooth varying states, eg stroma content.
      \end{itemize}
    \item Wrap-up.  Range of survival in PC is high, and is \emph{not} well-explained by known effects.  This is probably a consequence of the ad-hoc approach used to identify processes.  Thesis: that distinct molecular processes are determining survival in PC.  Set to find these using modern techniques.  Allude to future -- that these processes may point to new therapeutic directions and also are useful simply for staging (or maybe keep this point for the conclusion)
    \end{itemize}
  \end{itemize}
  
  \item Methods
  \begin{enumerate}
    \item Cohort recruitment and ethics (if this is the first survival chapter, else just reference the previous one)
    \item Sample prep and GEX wet lab
    \item Normalization, standardization, cohort subsetting, etc
    \item Filtering.  Unsupervised $\rightarrow$ Van-ISIS $\rightarrow$ AdaEN/SCAD/EN.  Select latter based on CV IBS, due to instability of AdaEN/SCAD.  What about TIE*?
    \item DR: Induce modules (APC) or metagenes (PCA).  Can investigate sparse PCA as intermediate method.
    \item Functional assignation.  Statnikov multiplicity problem.  GSVA or similar on MSigDB, also on sigs from above.  Correlate.
    \item Sanity checking.  Perform the above filtering and DR steps on permuted data.  Verify that the distribution of correlations with non-permuted signatures is as random as expected.
  \end{enumerate}
  
  \item Results
  \begin{enumerate}
    \item Cohort characteristics (if first surv chapter)
    \item Filtering metrics.  Unsup, ISIS, penalized.  Report CV IBS for penalized methods.
    \item DR.  Number of components / clusters with best CV IBS.
    \item Function.
  \end{enumerate}

  \item Conclusion
  
\end{itemize}
\end{document}
