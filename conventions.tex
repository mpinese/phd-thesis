\chapter*{Conventions}
Unless otherwise specified, the following conventions are used throughout this dissertation.

\begin{itemize}
  \item Indices in algorithm pseudocode are 1-based.
  \item Logarithms ($\log$) and exponentiations ($\exp$) are to base $e$.
  \item Square brackets around a predicate $P$ denote the Iverson bracket: $\left[P\right] \Leftrightarrow 1\ \mbox{if}\ P \mbox{ is true, else}\ 0$.
  \item Square brackets around a function-predicate pair $f(i)~|~P(i)$, indicate tuple builder notation: $\left[f(i)~|~P(i)\right]_{i=a}^b \Leftrightarrow \left[ f(a), f(a+1), \dots, f(b) \right]$, where an element $f(i)$ is only included in the tuple if $P(i)$ is true.
  \item $x_+$ indicates the value of the ramp function at real $x$, $x_+ := \max(0, x)$.
  \item $\mathbf{0}^n$ denotes the vector in $\mathbf{R}^n$ with all entries equal to zero.
  \item $\mathbb{B}$ denotes the Boolean set $\{true, false\}$.
\end{itemize}
