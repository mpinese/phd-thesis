Pancreatic ductal adenocarcinoma (PDAC) is the most deadly common cancer, with fewer than ten percent of patients surviving five years from diagnosis.  In contrast to many other cancers, the grim prognosis of PDAC has remained largely the same over the past thirty years, reflecting our relatively poor understanding of the disease.  This situation may soon change: recent large-scale efforts have produced detailed molecular and clinical data for almost a thousand cases of PDAC, finally enabling detailed dissection of the cancer's molecular basis.  This work used these new data to directly address two questions linked to PDAC's exceptionally poor prognosis: what are the biological processes that drive poor prognosis in PDAC? and can we develop a practical and effective prognostic staging system to better guide PDAC treatment?
\par
\Cref{chap:signatures} considered the first question, using a gene expression factorization approach to identify two metagenes that determine survival in PDAC.  These metagenes were linked to the biological processes of proliferation, and the EMT, and were also prognostic in a number of other solid tumours, revealing these processes as generally important to cancer patient survival.  
\par
\Cref{chap:nomogram} addressed the second question, developing a pilot tool to estimate the prognosis of a PDAC patient, and guide the decision of whether to resect that patient's tumour.  The tool is unique in that it can be applied prior to resection, and its validation performance was competitive with gold standard post-resection methods.  This performance might be improved through better selection of prognostic biomarkers; this is the subject of \cref{chap:messina}, which describes a method for the identification of biomarkers that are well-suited for development into clinical tests.
\par
PDAC has an exceptionally poor prognosis, but better understanding of the disease, and improvements in its management, can yield incrementally better outcomes.  The work presented here focussed on understanding and predicting prognosis in PDAC.  It identified two metagenes that were strongly linked to outcome, indicating valuable therapeutic directions to improve the survival of patients with particularly aggressive tumours; and it provided a proof-of-concept, and a method for development, of a biomarker-based preoperative prognostic tool for the improved management of PDAC.
