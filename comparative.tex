\documentclass[thesis.tex]{subfiles}
\begin{document}
\chapter{Comparative genomics}
\label{ch:comparative}

Outline ideas:
\begin{itemize}
  \item Introduction / overview:
  \begin{itemize}
    \item The use of models in PC (very brief)
    \item Specific models used in PC, with strong focus on the most common (KPC), and derivates.  Cover ease-of-use briefly.
    \item Current knowledge re: how appropriate the models are.  Consider histology, genetic features, disease progress (incl. metastatic potential), response to therapy.  Highlight gap in genetic information, and relevance to response to therapy.
    \item Brief overview of known genetic features of human disease.  Raise possibility of subtypes.
    \item Wrap-up with overview of project:
    \begin{enumerate}
      \item Collect matched tumour-normal DNA from a range of GEMMs.
      \item Sequence and determine conserved model-specific and general patterns of somatic mutation.
      \item Compare observed patterns to human disease.
      \begin{itemize}
        \item Are genetic features of human disease recapitulated generally in the models?
        \item Does a single model match the genetic features of human disease much better than the others?
        \item Do specific models serve as simulations of certain subtypes of human disease?
      \end{itemize}
    \end{enumerate}
    \item Overall thesis for this work: \\
    Matching patterns of genetic alterations in mouse models of pancreatic cancer to those seen in human disease can inform researchers as to which models are generally best, and which best match specific patient types. \\
    Sub-theses:
    \begin{itemize}
      \item The patterns of mutations seen in common mouse models of pancreatic cancer match those consistently seen in human disease.
      \item Different mouse models possess different mutation spectra, and models may be close fits to specific genetic subtypes of patients.
    \end{itemize}
  \end{itemize}
  
  \item Methods
  \begin{enumerate}
    \item Models
    \item Sample origin and processing
    \item Sequencing
    \item QC, mapping, realignment, BQSR
    \item Somatic SNV and indel detection and analysis (pathway)
    \item CNV and LOH detection
  \end{enumerate}
  
  \item Results
  \begin{enumerate}
    \item Somatic SNV and indels
    \item CNV and LOH
  \end{enumerate}

  \item Conclusion
  
\end{itemize}
\end{document}
