\documentclass[thesis.tex]{subfiles}
\begin{document}
\chapter{Comparative genomics}
\label{ch:comparative}

Outline ideas:
\begin{itemize}
  \item Introduction / overview:
  \begin{itemize}
    \item The use of models in PC (very brief)
    \item Specific models used in PC, with strong focus on the most common (KPC), and derivates.  Cover ease-of-use briefly.
    \item Current knowledge re: how appropriate the models are.  Consider histology, genetic features, disease progress (incl. metastatic potential), response to therapy.  Highlight gap in genetic information, and relevance to response to therapy.
    \item Brief overview of known genetic features of human disease.  Raise possibility of subtypes.
    \item Wrap-up with overview of project:
    \begin{enumerate}
      \item Collect matched tumour-normal DNA from a range of GEMMs.
      \item Sequence and determine conserved model-specific and general patterns of somatic mutation.
      \item Compare observed patterns to human disease.
      \begin{itemize}
        \item Are genetic features of human disease recapitulated generally in the models?
        \item Does a single model match the genetic features of human disease much better than the others?
        \item Do specific models serve as simulations of certain subtypes of human disease?
      \end{itemize}
    \end{enumerate}
    \item Overall thesis for this work: \\
    Matching patterns of genetic alterations in mouse models of pancreatic cancer to those seen in human disease can inform researchers as to which models are generally best, and which best match specific patient types. \\
    Sub-theses:
    \begin{itemize}
      \item The patterns of mutations seen in common mouse models of pancreatic cancer match those consistently seen in human disease.
      \item Different mouse models possess different mutation spectra, and models may be close fits to specific genetic subtypes of patients.
    \end{itemize}
  \end{itemize}
  
  \item Results
  \begin{enumerate}
    \item Somatic SNV and indels
    \item CNV and LOH
  \end{enumerate}

  \item Conclusion
  
\end{itemize}

\section{Methods}

\subsection{Models}

\subsection{Sample Origin and Processing}

\subsection{Sequencing}

\subsection{QC}

\subsection{Mapping}

For initial mapping, all lanes were processed independently.  SHRiMP was used to map colourspace reads to the mm10 genome using `all-contigs' and `single-best-mapping' options.  Unpaired reads in the source fastq files were mapped as single reads; paired reads were mapped with pair mode `opp-in', and a per-fastq insert size distribution estimated from a normal distribution fit to insert sizes of the first 10,000 reads.  Likely duplicate reads were marked using Picard MarkDuplicates on each individual lane BAM, using an optical duplicate pixel distance parameter of 10.

Lane BAMs were progressively merged: first, duplicate lane BAMs for a given mouse and sample type (tumour or normal) were combined, then tumour and normal BAMs for a given mouse, and finally combined tumour-normal BAMs for all mice.  Following each level of merging, GATK was used to separately perform local realignment and base quality score recalibration (LA-BQSR) on each BAM file of that level.  Finally, the full experiment BAM file was split by mouse and sample type for analysis, yielding 62 paired tumour and normal final BAMs.

\subsection{Somatic SNV and Indel Detection}

muTect and Strelka were used separately to detect somatic SNVs and indels in individual mouse tumour and normal BAMs.  muTect was supplied default parameters; Strelka used default parameter settings for the BWA mapper, modified so that isSkipDepthFilters = 1.  



\subsection{CNV and LOH Detection}

\end{document}
