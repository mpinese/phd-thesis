As tempted as I was at times to follow the path of Shivers' outstanding non-acknowledgement \cite{Shivers1994}, in truth this work would simply not be without the myriad contributions of friends, family, and colleagues.

I am indebted to Prof. Andrew Biankin, for teaching me some of the rules of professional science, and for his great accomplishment in piloting the \acrshort{APGI} to success; without the data generated by that project, this dissertation would not have been possible.  Likewise, Prof. Sean Grimmond and his group at the Institute of Molecular Bioscience were instrumental in generating the lion's share of the expression data upon which I worked.

Survival analysis requires exacting interpretation, coding, and follow-up of clinical records; laborious and difficult work that is too often left unacknowledged.  I am very grateful to the entire clinical data team of the \acrshort{APGI}/\acrshort{NSWPCN}, for thanklessly collating the high-quality data which I required, and for humouring my continual requests for data.  In particular, this work would be far weaker without the substantial contributions of Clare Watson, Skye Simpson, Mary-Anne Brancato, and Amber Johns.

Dr. David Chang taught me so much about true translational research; his work and guidance continues to transform the predictor of Chapter 3 from an academic exercise, into something that could truly make a positive difference for patients.  His contributions to the greater prognostic predictor project far outweigh mine, and his long efforts on molecular prognostics in pancreas cancer enabled the small work I describe here.

Dr. Mark Cowley, I suspect, was not quite aware of the task ahead when he kindly agreed to supervise me following Prof. Biankin's departure.  This dissertation is stronger for his critical input, and his patient help and encouragement at all times was far better than I think I deserved.

Friends and family have given support, both overt and covert, freely through the last four years.  In particular, Aaron Statham, Alison Ferguson, and Drs. Hugh French and Brian Gloss, have made contributions to my mental health -- usually administered at Darlo Bar -- that go beyond the call of duty.  I have also been very fortunate to have had the unquestioning support of the Pinese and Scott families.

Scholarships from the Australian government, UNSW Australia, and the Garvan Institute meant that I did not need to cut back on my food habit in order to meet my basic coffee needs.

Too many people have assisted this work, directly or indirectly, for me to list them all here.  To all of these, I apologise if I have omitted you -- remind me sometime, and I promise you a drink of your choice.  Some I suspect have been assisting in secret.  To those, you really should have seen this omission coming, but thanks all the same.

It's traditional to save the best to last; in this case it's also amusing.  To my truly amazing wife Emma: thank you.  Though you didn't write a word, I see this dissertation as much your work as mine.  Throughout this PhD, you have shown fortitude and compassion that is inspirational.  I probably will never know all the secret machinations that you undertook to make this dissertation happen -- at least not until I receive an itemized bill -- but I want you to know that they worked, and that this simply wouldn't have come together without you.  And to little Eva: you just might read this some day, so you may as well know that you were a very early overachiever -- you gave Dad a great big kick up the arse, even when you were less than two feet tall!

\vspace{15mm}

\hspace{70mm}Sydney

\hspace{70mm}23 March 2015