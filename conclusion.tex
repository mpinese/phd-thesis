\documentclass[thesis.tex]{subfiles}
\begin{document}
\chapter[Conclusion]{Conclusion}
\label{chap:conc}

\acrlong{PDAC} is a devastating cancer that is remarkably refractory to all current therapies, and there is a real need for better understanding and management of this disease.  Recent genomic efforts have begun to exhaustively characterise \gls{PDAC} at the molecular level, but the analysis of the data generated by these projects has only just begun.  This dissertation reported on one aspect of this analysis, by exploring the link between gene expression and \gls{PDAC} outcome, in ways that were previously not possible.  It yielded a better understanding of the basic biological processes that reflect, and possibly control, survival in \gls{PDAC}, and developed tools to assist the management of the disease, with the goal of ultimately improving patient quality of life.  Beyond the application to \gls{PDAC} presented here, the analysis frameworks used in this dissertation might find use in exploring the links between gene expression and prognosis in other diseases.

TODO: summary of signatures.  Cover:
  * Main result (2 sigs)
  * Secondary result (other hallmarks -- not so important in PDAC, at least not differentially)
  * Tertiary result (sigs useful in other cancers)
  * Segue to potential generality of technique for investigation of prognostic metagenes.  Cover briefly advantages over other methods -- refer back to NMF vs ICA vs PCA?  Also reflect on the difficulty of correct application of NMF.
  * Final summary: interesting results for PDAC, solid framework for investigation of prognostic metagenes in other diseases.

TODO: summary of nomogram.  Cover:
  * 
  
TODO: summary of Messina2

TODO: Final wrap-up para.

\end{document}
