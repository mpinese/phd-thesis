\documentclass[a4paper,12pt,stdletter,sigleft]{newlfm}
\namefrom{Mark Pinese}
\addrfrom{Kinghorn Cancer Centre \\ 370 Victoria St \\ Sydney  NSW  2010 }
\addrto{Graduate Research School \\ Kensington Campus \\ University of New South Wales  NSW  2052 }
\greetto{Dear Sir or Madam,}
\closeline{Yours Faithfully,}

\newlfmP{dateskipbefore=50pt}
\newlfmP{sigsize=50pt}
\newlfmP{sigskipbefore=50pt}
\newlfmP{Headlinewd=0pt,Footlinewd=0pt}

\leftmarginsize{3cm}
\rightmarginsize{3cm}

\begin{document}
\begin{newlfm}

Re: Response to examiners' reports on PhD dissertation, student number 3062573.

I thank the examiners for their time, and helpful comments on my dissertation.  Where possible, I have addressed the examiners' comments through revisions to the dissertation, which are documented in the following.  In cases where revision was not practical, I have also included my responses to the examiners' concerns in this letter.

\emph{Chapter 1} \\
The examiners did not detail any specific concerns with this chapter.

\emph{Chapter 2} \\
Examiner 1 asked whether a prognostic predictor could be suggested that uses fewer than the 361 gene expression measurements comprising the full PARSE score.  This is certainly possible, and I have added a brief note to the dissertation that outlines how such a smaller predictor could be developed by using the data in Appendix D.  I decided not to include such a smaller predictor in Chapter 2 of the dissertation, as this problem of identifying very small prognostic predictors is already directly addressed by Chapter 4.

Examiner 2 queried whether different survival-associated metagenes were found in different clinical subsets of the patient cohort.  Unfortunately, the number of patients within each subset was too small to permit this particular analysis.  However, for the two metagenes that were identified using the whole discovery cohort, metagene activity was verified to be independent of all measured clinico-pathological variables; this result is in the dissertation.

Examiner 2 asked for additional information on gene selection methods, and further discussion on the effect of this selection step on the metagenes identified; these have been added.  Examiner 2 also queried why both metagenes found were associated with decreased survival.  The analysis was not biased towards detecting metagenes associated with either better or worse survival, and with only two metagenes identified, it is my opinion that both of them being associated with poorer survival is coincidental.

Examiner 2 requested a comparison of the metagene activity scores to histopathological features, such as Ki-67 or E-cadherin staining.  Unfortunately, these features were only available for closely-matched tissue samples in a very small number of cases, making meaningful comparison impossible.  This is certainly a valuable future direction for this work, particularly given its potential for clinical translation, and I have added a brief discussion of this point to the dissertation.

\emph{Chapter 3} \\
Examiner 1 asked whether any attempts were made to build a binary classifier to identify patients likely to benefit from surgery.  This avenue of investigation had been explored, both by discretising the continuous predictions from the algorithms presented in Chapter 3, and by Decision Curve Analysis (DCA)\footnote{Vickers, A. J., Cronin, A. M., Elkin, E. B., \& Gonen, M. (2008). Extensions to decision curve analysis, a novel method for evaluating diagnostic tests, prediction models and molecular markers. BMC Medical Informatics and Decision Making, 8, 53}.  However, the work did not bear fruit, for the following reasons: it was difficult to achieve consensus between surveyed surgeons on the baseline (non-resected) survival of a patient; there was no consensus on the odds ratio required for DCA; and it was impossible to condense a full survival curve into a single metric for comparison.  Consequently, the final predictors described in Chapter 3 simply plot the likely outcome trajectories, and leave the nuanced and personal length versus quality of life decisions to the surgeon and patient.

\emph{Chapter 4} \\
Examiner 1 asked whether any validation of this chapter's findings by IHC has been performed.  This has not yet been done, but it is certainly an important next step to validate the clinical utility of the Messina algorithm, as well as the specific prognostic classifiers described this chapter.

Brief discussion on the putative biomarkers described in this chapter has been added, as requested by Examiner 2.

\emph{Chapter 5} \\

\emph{General} \\
All spelling and typographical errors have been corrected, and the chapter structure and figure sizes have been revised, as suggested by Examiner 2.


\end{newlfm}
\end{document}