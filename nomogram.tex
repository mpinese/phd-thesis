\documentclass[dissertation.tex]{subfiles}
\begin{document}

\chapter{A Preoperative Molecular Prognostic for Pancreas Cancer}
\label{chap:nomogram}

\emph{Thesis: A preoperative prognostic tool for pancreas cancer can be developed to discriminate good between and poor prognosis patients more reliably than current methods.}

\paragraph{Summary}
For those patients fortunate enough to be diagnosed with a resectable tumour, surgical removal of the primary cancer is the best first-line therapy for pancreas cancer.  However, the significant morbidity associated with pancreas cancer resection makes it cruicially important to only operate on the patients who stand to benefit from the procedure.  Identifying just those patients who will respond to resection remains a serious challenge in pancreas cancer treatment: current criteria to select patients for resection perform poorly, and consequently many patients will undergo a complex procedure, with serious effects on future quality of life, for little benefit.  Tumour biomarkers have the potential to dramatically refine morphology-based staging criteria by supplying a direct readout of tumour biology, and recent technological developments have enabled the preoperative measurement of tissue biomarkers in pancreas cancer.  The ability to measure pancreas cancer tissue biomarker levels preoperatively, combined with the enhanced information on disease state available from tissue biomarkers, finally enables the development of preoperative staging criteria that accurately identify pancreas cancer patients for resection.  This chapter details the development and validation of a two-biomarker preoperative prognostic tool for resectable pancreas cancer, that can be used to define new resectability criteria to accurately select patients who stand to benefit from primary tumour resection.


\section{Introduction}
For patients with a resectable tumour and no known metastases, surgical removal of the primary tumour is the current recommended first-line therapy for pancreas cancer, and the only intervention offering the realistic possibility of a cure \cite{Editors2015}.  However, pancreas cancer resection is a major procedure, with the potential for serious complications, morbidity, and reduced quality of life following recovery \cite{Ho2005}.  Due to the substantial negative effects of surgery, the decision of whether or not to perform curative-intent resection should balance the risks of surgery against its expected benefits, tailored to each individual case.

Unfortunately, current practice guidelines recommend that curative-intent surgery be offered to all metastasis-free patients with a resectable tumour, with no consideration of personal benefit \cite{Editors2015}.  This blanket approach to selecting patients for curative resection has proven to be highly inadequate.  Even following pathologically complete tumour removal and adjuvant chemotherapy, more than 70\% of current pancreas ductal carcinoma patients will relapse with, and ultimately succumb to, distant metastases \cite{Barugola2007}.  These occult metastases must have been present prior to removal of the primary tumour, yet were undetectable during initial investigations, and their presence means that any curative-intent resection was futile.  As a result, the majority of `curative' resections that are undertaken based on current selection criteria are performed on patients with occult metastases, have no hope of actually effecting a cure, and would not have been undertaken at all if the presence of the metastases had been known prior to surgery.  Better methods for selecting patients for resection are urgently needed.

A number of pancreas cancer grading and schemes and prognostic tools have been described, but inconsistent performance, or a reliance on information that can only be known post-operatively, limits their use in pre-operative decisions.  The level of serum \gls{CA-19-9} is a well-characterised biomarker of pancreas cancer, with high levels correlating with increased tumour burden, lower probability of resectability, increased post-resection recurrence, and worse prognosis \cite{Kim2011, Ballehaninna2012, Barugola2007, Lundin1994}.  \Gls{CA-19-9} levels are easily determined pre-operatively, but the use of this marker is complicated by a lack of consensus on threshold concentrations, the elevation of \gls{CA-19-9} levels by a number of conditions other than pancreas cancer, and the complete absence of this marker in approximately 10\% of the general population.  Additionally, although \gls{CA-19-9} levels are statistically associated with post-resection recurrence by distant metastasis, a low \gls{PPV} renders the biomarker unhelpful when deciding whether or not to resect \cite{Kim2011}.

The current standard prognostic tool for pancreas cancer is the \gls{MSKCC} nomogram \cite{Brennan2004}, which integrates a number of \glspl{CPV} to arrive at point estimates for survival post-resection.  Unfortunately, its clinical utility is small: as it relies on information that is only available following resection, the \gls{MSKCC} nomogram is only useful in a post-operative context, and cannot assist in pre-operative decisions to resect.  This severely limiting reliance on postoperative variables was made necessary by the fact that most classical prognostic factors in pancreas cancer (such as lymph node infiltration, or histological grade) can only be reliably measured following resection (TODO cite to prog factors summary).  Any prognostic tool for pancreas cancer that relies only on classical \glspl{CPV} will likely share this same reliance on post-operative variables; an effective pre-operative prognostic conversely will need to be based on measurements other than the classical \glspl{CPV}.
%The \gls{MSKCC} nomogram also provides relatively little information to the user, supplying only an aggregate risk score and survival probabilities at three fixed times, with no indication of prediction confidence or prognosis over the full disease course.  These technical limitations are a consequence of the simple construction of the \gls{MSKCC} nomogram, so chosen to facilitate hand calculation.  With the widespread availability and use of portable computers and smart phones today, limiting performance for the sake of allowing hand calculation is no longer necessary, and return of far richer prognostic information is straightforward.

Tissue biomarkers can provide an almost direct window on the cellular state of tumour cells, and thus have the potential to predict cell behaviour far more reliably than macroscopic \glspl{CPV}.  Given that most pancreas cancer patients who undergo curative resection quickly recur due to occult metastases, biomarkers of metastasis have the potential to identify those patients who are likely to already have occult metastatic disease at the time of surgery, and thus better inform the decision to resect.  Two such biomarkers of metastasis are the cancer cell levels of the \gls{EMT}-related S100A2 and S100A4 proteins, both of which are strongly predictive of outcome following resection, and appear to reflect the presence of a pro-metastatic invasive phenotype in the cancer \cite{Biankin2008, Tsukamoto2013, Lee2014}.  Unfortunately, these tissue biomarkers have to date only been assessed in bulk tissue samples collected during surgery, and their utility, or even measurability, in a pre-operative setting, is untested.

Recent techological developments have made possible the pre-operative measurement of tissue biomarkers as a part of \gls{EUS}, a routine diagnostic modality for pancreas cancer.  \Gls{IHCal} staining has been successfully performed on \gls{FNA} biopsies of pancreas neoplasms collected during \gls{EUS} \cite{Popescu2012, Salla2009, Stelow2005}, and in principle \gls{EUS}-\gls{FNA}-\acrshort{IHCry} could form the basis of a pre-operative biomarker measurement methodology for routine use in pancreas cancer diagnosis.  Although this proposed biomarker measurement approach is not currently in common use, it utilises only techniques that are commonly available in pancreas cancer treatment centres, and thus has the potential to be rapidly integrated into current diagnostic workflows, should biomarker measurements prove to be clinically useful.

The nexus of known biomarkers of metastatic behaviour, and new pre-operatively applicable techniques to measure these biomarkers, presents an opportunity to address the pressing need for better criteria to select patients for pancreas cancer resection.  As part of the \gls{APGI}, as well as other work, the group has collected tissue measurements of S100A2 and S100A4 biomarkers, and detailed patient follow-up, for a large number of cases of pancreas cancer from a range of independent cohorts.  These cases will be used to develop a prognostic predictive tool for outcome following resection, that uses tissue levels of S100A2 and S100A4 as major prognostic factors.  This tool will rely on biomarker measurements made on tissue collected during resection, and thus will not be directly applicable pre-operatively.  However, pilot study data will be used to show that levels of S100A2 and S100A4 measured by pre-operative \gls{EUS}-\gls{FNA}-\acrshort{IHCry} correlate well to tissue levels of the biomarkers measured on peri-operative specimens, and therefore that biomarker-based pre-operative prediction of patient outcome following surgery is possible in a clinical setting.

The majority of pancreas cancer resection procedures today are performed on patients who should never have been offered surgical resection at all.  These patients have undetected metastases at the time of surgery, and will derive little benefit from a major operation that will have serious impacts on quality of life.  Current tools for patient staging and estimation of prognosis are either ineffective at identifying patients at risk for occult metastases, or only applicable post-operatively, and thus cannot be used to inform the decision of whether or not to resect.  It may be possible to identify pre-operatively those patients with occult metastases, by examining levels of known tissue biomarkers of pancreas cancer metastatic potential.  This chapter will describe the development of a prognostic tool for pancreas cancer that makes use of these biomarkers to predict personalised outcome following surgery, and assist in making treatment decisions appropriate for each individual pancreas cancer patient.

TODO: Add Ca-19-9 to nomogram?  At least have it as a preoperative comparison?  Subset pts by threshold of 200 U/mL preoperative (or 163 U/mL?).

\section{Results}
Data from a large retrospectively-acquired cohort, the \gls{NSWPCN} cohort, were used to derive a prognostic predictor for survival of pancreas cancer patients who underwent curative-intent resection.  Discrimination and calibration of this predictor were then verified on TODO number independent surgical cohorts.  Data from an \gls{EUS}-\gls{FNA}-\gls{IHC} pilot study were then used to establish that pre-operatively assessed tissue biomarker levels reflected measurements from operative biopsies, and thus that the prognostic predictor developed here could be applied in a pre-operative decision setting.

\subsection{Prognostic variables and biomarkers}
As the aim was to develop a prognostic predictor that could be applied pre-operatively, only the small subset of traditional \glspl{CPV} that were deemed to be measurable prior to resection were considered.  These variables were: patient sex and age at diagnosis, tumour location (dichotomised as head of pancreas vs other location), and size of the tumour's longest pathological axis.  All but one of these variables is straightforward to measure pre-operatively, with the exception being tumour size.  In the training and validation sets, tumour size was measured post-operatively from the resected tumour, whereas in pre-operative application tumour size must be estimated by imaging techniques such as \gls{CT} or \gls{EUS}.  The correlation between \gls{CT} and \gls{EUS} estimates of tumour size, and actual size upon resection, is strong but not perfect \cite{Arvold2011}.  As no pre-operative measurements of tumour size were available in the cohorts used for this work, pre-operative size was approximated by the post-operative measurement.  The implications of this approximation for the prognostic tool developed here, as well as for future work, are considered in the discussion.

The dichotomised tissue levels of two pre-operatively assessable biomarkers, S100A2 and S100A4, were considered when building the prognostic tool.  These biomarker levels were measured by \gls{IHC} on tissue collected during resection, and thus 

Scores for two pre-operatively assessable biomarkers, S100A2 and S100A4, were considered during prognostic development.  Post-operative

TODO: Biomarkers.  Specifically, why I didn't use CA-19-9.

\subsection{Cohorts and Characteristics}
General characteristics of the \gls{NSWPCN}, TODO, and TODO cohorts are summarised in \tref{TODO}.

\subsection{Exploratory Analysis}
Exploratory analyses were performed on the training cohort only, in order to ascertain the best model form for later fits.

\paragraph{Model functional form and expanded terms}
The \gls{CPH} framework was used to assess functional form for the two continuous covariates: age at diagnosis, and maximum pathological axis size.  \gls{LOESS} smooths of martingale residuals indicated a possible weak U-shaped relationship for age at diagnosis (TODO figref), and a knee-shaped form for size (TODO figref).  In subsequent fits these forms were modelled by adding a quadratic term for centered age at diagnosis, and a $size:I(size \leq knee)$ interaction for centered axis size.  The original set of five linear prognostic terms, plus the two additional nonlinear terms, was denoted the expanded term set.

\paragraph{Proportional hazards assumption}
A Grambsch-Therneau test \cite{Grambsch1994} on the \gls{CPH} model fit using all expanded terms indicated that patient sex violated the \glspl{PH} assumption (P = TODO) -- in other words, the two sexes had significantly different baseline hazard forms.  To account for this effect, all subsequent models were stratified by patient sex, so that the survival of male and female patients was modelled by two different baseline hazard functions.  A repeated Grambsch-Thernau test on the stratified model indicated no further significant violations of \gls{PH}.

\paragraph{Variable selection}
Genetic selection was used to identify the model with optimal \gls{BIC} from the set of all \gls{CPH} models that use any combination of the expanded terms.  Models with interactions between terms of up to degree two were considered, and a marginal term constraint was enforced, to ensure that interaction effects were only present in the model specification if the associated main effects were also.  Stratification of baseline hazard by patient sex was used in all models.  The identified optimal model used the three variables: tumour size (linear term only), S100A2 status, and S100A4 status, in addition to the sex stratum.  This model was also identified by stepwise backward selection for optimal BIC, starting from the marginal \gls{CPH} fit using all expanded terms.  The final \gls{BIC}-selected set of prognostic terms (tumour size linear term, S100A2 status, S100A4 status, and a patient sex stratum) was denoted the reduced term set.

\paragraph{Fit diagnostics}
The \gls{PH} assumption was again verified by the Grambsch-Therneau test on a \gls{CPH} fit to the reduced term set (global P = TODO).  Model residuals were inspected to identify possible outlier patients, and assess overall fit stability.  Deviance residuals indicated no egregious outlier patients, and $DFBETAS$ indicated TODO influential patients with $|DFBETAS_{i,j}| > \frac{2}{\sqrt{n}}$, where $n = TODO$.  TODO: Stuff on these outliers.

\paragraph{Summary}
Exploratory analysis on the training \gls{NSWPCN} cohort data indicated that a BIC-optimal \gls{CPH} fit was achieved by the following model: $Survival ~ Age + S100A2 + S100A4 + strata(Sex)$.  There was no significant evidence for nonlinear effects in any examined variables, or for any interactions up to degree two.  Male and female patients appeared to have significantly different baseline hazards, a feature that was modelled by stratifiying fits by patient sex.

\subsection{Final fits}


\subsection{Web tool}

* Summary of analysis
* Variables that are both available and reasonably pre-operatively assessable:
  # Patient:   Age, Sex
  # Imaging:   Size, Location (head vs tail)
  # Biomarker: A2, A4, CA-19-9 (thresh 100 U/mL)
* Cohort subsetting and characteristics (all of them)
* Exploration.
  # Functional forms.  Martingale residual smooths of marginal model.
  # Perform CPH fit of marginal model using identified forms, examine violations of CPH.  Identify Sex and A4 as violators.  Re-fit with strata to resolve.  Probably just describe this step as a reason for stratification by Sex, A4.
* All-subsets search for best model, including interactions.


TODO: In app, show +/- margin curves, to guide surgeons as to benefit from aggressive surgery.


\subsection{Cohort characteristics}
\label{subsec:nomo-results-cohort}

\subsection{Development of a preoperative prognostic model}

\subsection{Validation of the prognostic model}
Intro here on disc \& calib.  Also describe cohorts briefly.
\subsubsection{Discrimination}
\subsubsection{Calibration}

\subsection{Web tool TODO -- better name}

\section{Discussion}

\section{Methods}
\subsection{Cohort recruitment and ethics}
\label{subsec:nomo-methods-cohort}

\end{document}
