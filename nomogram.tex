\documentclass[dissertation.tex]{subfiles}
\begin{document}

\chapter{A Molecular Prognostic Nomogram for Pancreas Cancer}
\label{chap:nomogram}

\emph{Thesis: A preoperative prognostic tool for pancreas cancer can be developed to discriminate good between and poor prognosis patients more reliably than current methods.}

\paragraph{Summary}
For those patients fortunate enough to be diagnosed with a resectable tumour, surgical removal of the primary cancer is the best first-line therapy for pancreas cancer.  However, the significant morbidity associated with pancreas cancer resection makes it cruicially important to only operate on the patients who stand to benefit from the procedure.  Identifying just those patients who will respond to resection remains a serious challenge in pancreas cancer treatment: current criteria to select patients for resection perform poorly, and consequently many patients will undergo a serious procedure, with serious side effects, for little benefit.  Tumour biomarkers have the potential to dramatically refine morphology-based staging criteria by supplying a direct readout of tumour biology, and recent technological developments have enabled the preoperative measurement of tissue biomarkers in pancreas cancer.  The ability to measure pancreas cancer tissue biomarker levels preoperatively, combined with the enhanced information on disease state available from tissue biomarkers, finally enables the development of preoperative staging criteria that accurately identify pancreas cancer patients for resection.  This chapter details the development and validation of a two-biomarker preoperative prognostic tool for resectable pancreas cancer, that can be used to define new resectability criteria, and accurately select patients for primary tumour resection.

\section{Introduction}


Pancreas cancer: surgery as the current front-line therapy.  However, resection is a major procedure with significant comorbidity.  There is therefore a need to balance the benefit from surgery against the harm from it.

DC: The pattern of failure is primarily metastatic disease, with fewer than 30\% of patients having local recurrence only. This suggests that occult metastatic disease was present at the time of surgery which was not detected using current staging modalities. Hence better methods are urgently needed to select patients for pancreaticoduodenectomy.

\section{Results}
\subsection{Cohort characteristics}
\label{subsec:nomo-results-cohort}

\subsection{Development of a preoperative prognostic model}

\subsection{Validation of the prognostic model}
Intro here on disc & calib.  Also describe cohorts briefly.
\subsubsection{Discrimination}
\subsubsection{Calibration}

\subsection{Web tool TODO -- better name}

\section{Discussion}

\section{Methods}
\subsection{Cohort recruitment and ethics}
\label{subsec:nomo-methods-cohort}

\end{document}
