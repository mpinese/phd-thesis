\documentclass[dissertation.tex]{subfiles}
\begin{document}

\chapter{A Preoperative Molecular Prognostic for Pancreas Cancer}
\label{chap:nomogram}

\emph{Thesis: A preoperative prognostic tool for pancreas cancer can be developed to discriminate good between and poor prognosis patients more reliably than current methods.}

\paragraph{Summary}
For those patients fortunate enough to be diagnosed with a resectable tumour, surgical removal of the primary cancer is the best first-line therapy for pancreas cancer.  However, the significant morbidity associated with pancreas cancer resection makes it cruicially important to only operate on the patients who stand to benefit from the procedure.  Identifying just those patients who will respond to resection remains a serious challenge in pancreas cancer treatment: current criteria to select patients for resection perform poorly, and consequently many patients will undergo a complex procedure, with serious effects on future quality of life, for little benefit.  Tumour biomarkers have the potential to dramatically refine morphology-based staging criteria by supplying a direct readout of tumour biology, and recent technological developments have enabled the preoperative measurement of tissue biomarkers in pancreas cancer.  The ability to measure pancreas cancer tissue biomarker levels preoperatively, combined with the enhanced information on disease state available from tissue biomarkers, finally enables the development of preoperative staging criteria that accurately identify pancreas cancer patients for resection.  This chapter details the development and validation of a two-biomarker preoperative prognostic tool for resectable pancreas cancer, that can be used to define new resectability criteria to accurately select patients who stand to benefit from primary tumour resection.


\section{Introduction}
For patients with a resectable tumour and no known metastases, surgical removal of the primary tumour is the current recommended first-line therapy for pancreas cancer, and the only intervention offering the realistic possibility of a cure \cite{Editors2015}.  However, pancreas cancer resection is a major procedure, with the potential for serious complications, morbidity, and reduced quality of life following recovery \cite{Ho2005}.  Due to the substantial negative effects of surgery, the decision of whether or not to perform curative-intent resection should balance the risks of surgery against its expected benefits, tailored to each individual case.

Unfortunately, current practice guidelines recommend that curative-intent surgery be offered to all metastasis-free patients with a resectable tumour, with no consideration of personal benefit \cite{Editors2015}.  This blanket approach to selecting patients for curative resection has proven to be highly inadequate.  Even following pathologically complete tumour removal and adjuvant chemotherapy, more than 70\% of current pancreas ductal carcinoma patients will relapse with, and ultimately succumb to, distant metastases \cite{Barugola2007}.  These occult metastases must have been present prior to removal of the primary tumour, yet were undetectable during initial investigations, and their presence means that any curative-intent resection was futile.  As a result, the majority of `curative' resections that are undertaken based on current selection criteria are performed on patients with occult metastases, have no hope of actually effecting a cure, and would not have been undertaken at all if the presence of the metastases had been known prior to surgery.  Better methods for selecting patients for resection are urgently needed.

A number of pancreas cancer grading and schemes and prognostic tools have been described, but inconsistent performance, or a reliance on information that can only be known post-operatively, limits their use in pre-operative decisions.  The level of serum \gls{CA-19-9} is a well-characterised biomarker of pancreas cancer, with high levels correlating with increased tumour burden, lower probability of resectability, increased post-resection recurrence, and worse prognosis \cite{Kim2011, Ballehaninna2012, Barugola2007, Lundin1994}.  \Gls{CA-19-9} levels are easily determined pre-operatively, but the use of this marker is complicated by a lack of consensus on threshold concentrations, the elevation of \gls{CA-19-9} levels by a number of conditions other than pancreas cancer, and the complete absence of this marker in approximately 10\% of the general population.  Additionally, although \gls{CA-19-9} levels are statistically associated with post-resection recurrence by distant metastasis, a low \gls{PPV} renders the biomarker unhelpful when deciding whether or not to resect \cite{Kim2011}.

The current standard prognostic tool for pancreas cancer is the \gls{MSKCC} nomogram \cite{Brennan2004}, which integrates a number of \glspl{CPV} to arrive at point estimates for survival post-resection.  Unfortunately, its clinical utility is small: as it relies on information that is only available following resection, the \gls{MSKCC} nomogram is only useful in a post-operative context, and cannot assist in pre-operative decisions to resect.  This severely limiting reliance on postoperative variables was made necessary by the fact that most classical prognostic factors in pancreas cancer (such as lymph node infiltration, or histological grade) can only be reliably measured following resection (TODO cite to prog factors summary).  Any prognostic tool for pancreas cancer that relies only on classical \glspl{CPV} will likely share this same reliance on post-operative variables; an effective pre-operative prognostic conversely will need to be based on measurements other than the classical \glspl{CPV}.
%The \gls{MSKCC} nomogram also provides relatively little information to the user, supplying only an aggregate risk score and survival probabilities at three fixed times, with no indication of prediction confidence or prognosis over the full disease course.  These technical limitations are a consequence of the simple construction of the \gls{MSKCC} nomogram, so chosen to facilitate hand calculation.  With the widespread availability and use of portable computers and smart phones today, limiting performance for the sake of allowing hand calculation is no longer necessary, and return of far richer prognostic information is straightforward.

Tissue biomarkers can provide an almost direct window on the cellular state of tumour cells, and thus have the potential to predict cell behaviour far more reliably than macroscopic \glspl{CPV}.  Given that most pancreas cancer patients who undergo curative resection quickly recur due to occult metastases, biomarkers of metastasis have the potential to identify those patients who are likely to already have occult metastatic disease at the time of surgery, and thus better inform the decision to resect.  Two such biomarkers of metastasis are the cancer cell levels of the \gls{EMT}-related S100A2 and S100A4 proteins, both of which are strongly predictive of outcome following resection, and appear to reflect the presence of a pro-metastatic invasive phenotype in the cancer \cite{Biankin2008, Tsukamoto2013, Lee2014}.  Unfortunately, these tissue biomarkers have to date only been assessed in bulk tissue samples collected during surgery, and their utility, or even measurability, in a pre-operative setting, is untested.

Recent techological developments have made possible the pre-operative measurement of tissue biomarkers as a part of \gls{EUS}, a routine diagnostic modality for pancreas cancer.  \Gls{IHCal} staining has been successfully performed on \gls{FNA} biopsies of pancreas neoplasms collected during \gls{EUS} \cite{Popescu2012, Salla2009, Stelow2005}, and in principle \gls{EUS}-\gls{FNA}-\acrshort{IHCry} could form the basis of a pre-operative biomarker measurement methodology for routine use in pancreas cancer diagnosis.  Although this proposed biomarker measurement approach is not currently in common use, it utilises only techniques that are commonly available in pancreas cancer treatment centres, and thus has the potential to be rapidly integrated into current diagnostic workflows, should biomarker measurements prove to be clinically useful.

The nexus of known biomarkers of metastatic behaviour, and new pre-operatively applicable techniques to measure these biomarkers, presents an opportunity to address the pressing need for better criteria to select patients for pancreas cancer resection.  As part of the \gls{APGI}, as well as other work, the group has collected tissue measurements of S100A2 and S100A4 biomarkers, and detailed patient follow-up, for a large number of cases of pancreas cancer from a range of independent cohorts.  These cases will be used to develop a prognostic predictive tool for outcome following resection, that uses tissue levels of S100A2 and S100A4 as major prognostic factors.  This tool will rely on biomarker measurements made on tissue collected during resection, and thus will not be directly applicable pre-operatively.  However, pilot study data will be used to show that levels of S100A2 and S100A4 measured by pre-operative \gls{EUS}-\gls{FNA}-\acrshort{IHCry} correlate well to tissue levels of the biomarkers measured on peri-operative specimens, and therefore that biomarker-based pre-operative prediction of patient outcome following surgery is possible in a clinical setting.

The majority of pancreas cancer resection procedures today are performed on patients who should never have been offered surgical resection at all.  These patients have undetected metastases at the time of surgery, and will derive little benefit from a major operation that will have serious impacts on quality of life.  Current tools for patient staging and estimation of prognosis are either ineffective at identifying patients at risk for occult metastases, or only applicable post-operatively, and thus cannot be used to inform the decision of whether or not to resect.  It may be possible to identify pre-operatively those patients with occult metastases, by examining levels of known tissue biomarkers of pancreas cancer metastatic potential.  This chapter will describe the development of a prognostic tool for pancreas cancer that makes use of these biomarkers to predict personalised outcome following surgery, and assist in making treatment decisions appropriate for each individual pancreas cancer patient.

TODO: Add Ca-19-9 to nomogram?  At least have it as a preoperative comparison?  Subset pts by threshold of 200 U/mL preoperative (or 163 U/mL?).

\section{Results}
\subsection{Cohort characteristics}
\label{subsec:nomo-results-cohort}

\subsection{Development of a preoperative prognostic model}

\subsection{Validation of the prognostic model}
Intro here on disc \& calib.  Also describe cohorts briefly.
\subsubsection{Discrimination}
\subsubsection{Calibration}

\subsection{Web tool TODO -- better name}

\section{Discussion}

\section{Methods}
\subsection{Cohort recruitment and ethics}
\label{subsec:nomo-methods-cohort}

\end{document}
