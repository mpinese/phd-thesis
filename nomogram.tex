\documentclass[dissertation.tex]{subfiles}
\begin{document}

\chapter{A Preoperative Molecular Prognostic for Pancreas Cancer}
\label{chap:nomogram}

\emph{Thesis: A preoperative prognostic tool for pancreas cancer can be developed to discriminate good between and poor prognosis patients more reliably than current methods.}

\paragraph{Summary}
For those patients fortunate enough to be diagnosed with a resectable tumour, surgical removal of the primary cancer is the best first-line therapy for pancreas cancer.  However, the significant morbidity associated with pancreas cancer resection makes it cruicially important to only operate on the patients who will derive a net benefit from the procedure.  Identifying just those patients who will respond to resection remains a serious challenge in pancreas cancer treatment: current criteria to select patients for resection perform poorly, and consequently many patients undergo a complex procedure, with serious effects on future quality of life, for little benefit.  Tumour biomarkers have the potential to dramatically refine morphology-based staging criteria by supplying a direct readout of tumour biology, and recent technological developments have enabled the preoperative measurement of tissue biomarkers in pancreas cancer.  The ability to measure pancreas cancer tissue biomarker levels preoperatively, combined with the enhanced information on disease state available from tissue biomarkers, finally enables the development of preoperative staging systems that accurately identify pancreas cancer patients for resection.  This chapter details the development and validation of \gls{PCOP}\glsreset{PCOP}, a two-biomarker prognostic tool for resectable pancreas cancer, that is in principle pre-operatively assessable, and can assist in making personalised treatment decisions.


\section{Introduction}
For patients with a resectable tumour and no known metastases, surgical removal of the primary tumour is the current recommended first-line therapy for pancreas cancer, and the only intervention offering the realistic possibility of a cure \cite{Editors2015}.  However, pancreas cancer resection is a major procedure, with the potential for serious complications, morbidity, and reduced quality of life following recovery \cite{Ho2005}.  Due to the substantial negative effects of surgery, the decision of whether or not to perform curative-intent resection should balance the risks of surgery against its expected benefits, for each individual case.

Unfortunately, current practice guidelines recommend that curative-intent surgery be offered to all metastasis-free patients with a resectable tumour, with no consideration of personal benefit \cite{Editors2015}.  This blanket approach to selecting patients for curative resection has proven to be highly inadequate.  Even following pathologically complete tumour removal and adjuvant chemotherapy, more than 70\% of current pancreas ductal carcinoma patients will relapse with, and ultimately succumb to, distant metastases \cite{Barugola2007}.  These occult metastases must have been present prior to removal of the primary tumour, yet were undetectable during initial investigations, and their presence means that any curative-intent resection was futile.  As a result, the majority of `curative' resections that are undertaken based on current selection criteria are performed on patients with occult metastases, have no hope of actually effecting a cure, and would not have been undertaken at all if the presence of metastatic disease had been known prior to surgery.  Better methods to select patients for resection are urgently needed.

The decision of whether or not to resect will always involve close consultation between the patient and doctors, comparing the costs and benefits of surgery as appropriate to each patient's case.  The downsides of resection are well-understood, but the benefit to be gained is challenging to quantify and communicate, being highly dependent on the many particulars of an individual patient's disease.  A simple approach to represent benefit from pancreas cancer resection involves survival curves, which plot the probability that a patient will be alive, at a range of times following resection (TODO example fig).  A survival curve distills the information from an arbitrary number of prognostic factors into a single simple figure, to provide an intuitive overview of a patient's expected disease course.  If accurate, such curves can provide a concrete measure of benefit from resection, and thus provide invaluable input into the treatment decision process.  Survival curves can be involved to calculate, which has likely limited their historical use.  However, the modern prevalence of computers removes this barrier, and even a modest device such as a smart phone can easily run prognostic tools capable of generating accurate personalised patient survival curves.

A number of pancreas cancer grading and schemes and prognostic tools have been described, but inconsistent performance, or a reliance on information that can only be known post-operatively, limits their use in pre-operative decisions.  The level of serum \gls{CA-19-9} is a well-characterised biomarker of pancreas cancer, with high levels correlating with increased tumour burden, lower probability of resectability, increased post-resection recurrence, and worse prognosis \cite{Kim2011, Ballehaninna2012, Barugola2007, Lundin1994}.  \Gls{CA-19-9} levels are easily determined pre-operatively, but the use of this marker is complicated by a lack of consensus on threshold concentrations, the elevation of \gls{CA-19-9} levels by a number of conditions other than pancreas cancer, and the complete absence of this marker in approximately 10\% of the general population \cite{Ballehaninna2012}.  Additionally, although \gls{CA-19-9} levels are statistically associated with post-resection recurrence by distant metastasis, a very low \gls{PPV} renders the biomarker unhelpful when deciding whether or not to resect \cite{Kim2011}.

The current standard prognostic tool for pancreas cancer is the \gls{MSKCC} nomogram \cite{Brennan2004}, which integrates a number of \glspl{CPV} to arrive at point estimates of survival post-resection.  Unfortunately, its clinical utility is small: as it relies on information that is only available following resection, the \gls{MSKCC} nomogram is only useful in a post-operative context, and cannot assist in pre-operative decisions to resect.  This severely limiting reliance on postoperative variables is made necessary by the fact that all strong classical prognostic factors in pancreas cancer (such as lymph node infiltration, resection margin status, or histological grade \cite{Bilici2014}) can only be reliably measured following resection.  Any prognostic tool for pancreas cancer that relies heavily on classical \glspl{CPV} will very likely share this same reliance on post-operative variables, and so an effective pre-operatively assessable prognostic will need to shirk classical \glspl{CPV}, and leverage novel pre-operative measures of prognosis.

Levels of tissue biomarkers directly reflect cellular state, and thus have the potential to predict cancer behaviour far more reliably than macroscopic \glspl{CPV}.  Given that most pancreas cancer patients who undergo curative resection quickly recur due to occult metastases, biomarkers of metastasis have the potential to identify those patients who are likely to already have occult metastatic disease at the time of surgery, and thus better inform the decision to resect.  Two such biomarkers of metastasis are the cancer cell levels of the \gls{EMT}-related S100A2, and S100A4 proteins, both of which are strongly predictive of outcome following resection, and appear to reflect the presence of a pro-metastatic invasive phenotype in the cancer \cite{Biankin2008, Tsukamoto2013, Lee2014}.  Despite this promise, these tissue biomarkers have to date only been assessed in bulk tissue samples collected during surgery, and their utility, or even measurability, in a pre-operative setting, is untested.

Recent techological developments have made possible the pre-operative measurement of tissue biomarkers during \gls{EUS}, a routine diagnostic modality for pancreas cancer.  \Gls{IHCal} staining has been successfully performed on \gls{FNA} biopsies of pancreas neoplasms collected during \gls{EUS} \cite{Popescu2012, Salla2009, Stelow2005}, and in principle \gls{EUS}-\gls{FNA}-\acrshort{IHCry} could form the basis of a routine pre-operative biomarker measurement methodology in pancreas cancer.  This proposed biomarker measurement approach utilises only techniques that are commonly available in pancreas cancer treatment centres, and thus has the potential to be rapidly integrated into current diagnostic workflows, should biomarker measurements prove to be clinically valuable.

The nexus of known biomarkers of metastatic behaviour, new pre-operatively applicable techniques to measure these biomarkers, and multiple large, clinically annotated cohorts of resected pancreas cancer, presents an opportunity to address the pressing need for better criteria to select patients for pancreas cancer resection.  As part of the \gls{APGI}, as well as other work, the group has collected tissue measurements of S100A2 and S100A4 biomarkers, and detailed patient follow-up, for a large number of cases of pancreas cancer from a range of independent cohorts.  These cases will be used to develop \gls{PCOP}, a tool to predict outcome following resection, using tissue levels of S100A2 and S100A4 as major prognostic factors.  This initial version of \gls{PCOP} is based on biomarker measurements made on tissue collected during resection, and thus will not be directly applicable pre-operatively.  However, pilot study data will be used to demonstrate that levels of S100A2 and S100A4 measured by pre-operative \gls{EUS}-\gls{FNA}-\acrshort{IHCry} correlate well to tissue levels of the biomarkers measured on operative specimens, indicating that a more refined version of \gls{PCOP} trained on pre-operative data will be equally effective.

The majority of pancreas cancer resection procedures today are performed on patients who should never have been offered surgical resection at all.  These patients have undetected metastases at the time of surgery, and will derive little benefit from a major operation, that has serious impacts on quality of life.  Current tools for patient staging and estimation of prognosis are either ineffective at identifying patients at risk for occult metastases, or only applicable post-operatively, and so cannot be used to inform the decision of whether or not to resect.  Tissue biomarkers of metastatic potential might identify, pre-operatively, those patients who have a high likelihood of metastatic disease, greatly assisting disease management decisions.  This metastasis prediction can be integrated with other clinical variables to yield personalised estimates of prognosis over time, that are well-suited to .  This chapter describes the use of pre-operatively assessable variables, including biomarker measurements, to create \gls{PCOP}, a tool that produces estimates of prognosis.  \Gls{PCOP} provides a natural way to show the influence of risk factors on a patient's personalised prognostic path, and thus can assist in making treatment decisions appropriate for each individual pancreas cancer patient.

\section{Results}
Data from the large, retrospectively-acquired \gls{NSWPCN} cohort were used to derive \gls{PCOP}, a tool to predict the survival of pancreas cancer patients following curative-intent resection.  Discrimination and calibration of \gls{PCOP} were verified on two independent surgical cohorts.  Data from an \gls{EUS}-\gls{FNA}-\acrshort{IHCry} pilot study established that pre-operatively assessed tissue biomarker levels reflected measurements from operative biopsies, and therefore that \gls{PCOP} could be translated to a pre-operative decision setting.

\subsection{Prognostic variables and biomarkers}
As the aim was to develop a prognostic predictor that could be applied pre-operatively, only factors that could be practically measured pre-operatively were considered for inclusion in the \gls{PCOP}.  The traditional \glspl{CPV} that were judged to be pre-operatively assessable were patient sex, patient age at diagnosis, tumour location (dichotomised as head of pancreas vs other location), and size of the tumour's longest pathological axis.  In addition to these traditional factors, the dichotomised tissue levels of S100A2 and S100A4 proteins were included as candidate biomarkers in the construction of the \gls{PCOP}.  Pre-operative blood levels of the biomarker \gls{CA-19-9} were available for a subset of the training cohort, but none of the validation sets; for this reason, and the marker's generally poor performance in isolation \cite{Kim2011}, \gls{CA-19-9} levels were not considered for inclusion in the \gls{PCOP}.

Pre-operative measurements of tumour size (for example, by \gls{CT} X-ray or \gls{EUS}) were not available in the training and validation sets, and were approximated by post-operative measurements during the development and testing of this nomogram.  Similarily, biomarker measurements were approximated using \gls{IHCal} staining of tissue collected during resection, as only very limited pre-operative \gls{EUS}-\gls{FNA} samples were available in the cohorts used.  The implications of these approximations for the prognostic tool developed here, as well as for future work, are considered in the discussion.  

\mpfatal{For the disc: Although the correlation between \gls{CT} and \gls{EUS} estimates of tumour size, and actual size upon resection, is respectable \cite{Arvold2011}, full clinical validation of this prognostic's use in a pre-operative setting will require ...}

\subsection{Cohorts and Characteristics}
General characteristics of the \gls{NSWPCN}, Glasgow, and Dresden cohorts are summarised in \tref{tab:nomo-cohort-characteristics}.  The \gls{NSWPCN} training cohort contained a small subgroup of patients with abnormally long recorded survival times ($> 3000$ days, 7/256 patients), that were strongly suspected to represent data errors, either as a consequence of incorrect coding following loss to follow-up, or misdiagnosis.  Given the age of the cohort, it was deemed impractical to revisit the original records to check these patients, and so all patients with recorded survival times exceeding 3000 days were excluded from the \gls{NSWPCN} training data.  The \gls{NSWPCN} cohort characteristics in \tref{tab:nomo-cohort-characteristics} have been calculated on the 249 patients that passed the 3000 day data quality cutoff.

\begin{table}[h]
\centering
\caption[Characteristics of patient cohorts]{Characteristics of the \gls{NSWPCN} training cohort, and the APGI, Dresden, and Glasgow validation cohorts.  Ordinal variables are shown as median, with quartiles in parentheses.  Categorical variables for which percentages do not add up to 100\% indicate the presence of minor unlisted categories.}\label{tab:nomo-cohort-characteristics}
\resizebox{\textwidth}{!}{
\begin{tabular}{llllll}
\toprule
                       &         & Training      & \multicolumn{3}{c}{Validation} \\ \cmidrule(r){4-6}
Characteristic         &         & NSWPCN        & APGI         & Dresden      & Glasgow        \\ \midrule
Number of patients     &         & 249           & 75           & xx           & 189            \\
Gender                 & Male    & 49.4\%        & 54.7\%       & xx\%         & 52.9\%         \\
Tumour location        & Head    & 80.7\%        & 85.3\%       & xx\%         & 100\%          \\
Excision margin status & R0      & 58.2\%        & 32.0\%       & xx\%         & 27.0\%         \\
Node involvement       &         & 65.8\%        & 78.7\%       & xx\%         & 82.5\%         \\
S100A2 positive        &         & 16.1\%        & 14.7\%       & xx\%         & 32.8\%         \\
S100A4 positive        &         & 75.5\%        & 65.3\%       & xx\%         & 70.9\%         \\
Disease-specific death &         & 95.2\%        & 68.0\%       & xx\%         & 85.2\%         \\[3]
Size of longest axis   & (mm)    & 30            & 35           & xx           & 30             \\
                       &         & (25 - 40)     & (28 - 43)    & (xx - xx)    & (25 - 40)      \\[3]
Age at diagnosis       & (years) & 69            & 67           & xx           & 64.0           \\
                       &         & (62 - 75)     & (61 - 74)    & (xx - xx)    & (57.8 - 69.4)  \\[3]
Length of follow-up    & (days)  & 479           & 655          & xx           & 501            \\
                       &         & (270 - 851)   & (362 - 743)  & (xx - xx)    & (233 - 915)    \\ \bottomrule
\end{tabular}
}
\end{table}

\subsection{Prognostic model building and selection}
Candidate prognostic models were constructed on the \gls{NSWPCN} training data by iterative model fitting, evaluation, and refinement.  To guard against overfitting caused by this iterative process, the \gls{NSWPCN} cohort was randomly split once into model building and testing sets.  All model fitting and refinement described below was performed on the 200-patient model building set, to yield three final candidate prognostic predictors.  The performance of each of these three predictors was then assessed on the 49-patient model test set, and the most parsimonious high-performing model was chosen as the final prognostic predictor, for subsequent external validation.

\paragraph{Cohort shift}
The \gls{NSWPCN} training cohort was collected over a long period, with patient diagnosis dates spanning the thirteen years from 1994 to 2006.  Over such an extended interval, subtle changes in cohort composition or therapy may cause a shift in cohort characteristics, and reduce the prognostic performance of a model that was built on the historical data, when it is applied to contemporary cases.  Cohort shift was investigated by examining the association between date of diagnosis, and all prognostic and outcome variables: in the absence of shift, no variables would be expected to change significantly over time.  Date of diagnosis was not significantly associated with any other variable (7 tests, lowest $P = 0.35$); there was therefore no indication of cohort shift in the \gls{NSWPCN} training data.

\paragraph{Model functional form and expanded terms}
The \gls{CPH} framework was used to assess functional form for the two continuous covariates: age at diagnosis, and maximum pathological axis size.  \gls{LOESS} smooths of martingale residuals \cite{Therneau1990} indicated a largely linear relationship for age at diagnosis (\fref{fig:nomo-funcform-age}), and a knee-shaped form for size (\fref{fig:nomo-funcform-size}), with the knee at approximately $0$ in median-centered units.  In subsequent fits this potential nonlinear size effect was modelled by adding a $\mbox{size}_+$ ramp term.  The original set of five linear prognostic terms, plus the additional nonlinear size term, was denoted the expanded term set.

\begin{figure}
\centering
  \subbottom[Patient age (centered)]{
    \includegraphics[width=.45\linewidth]{analysis/nomogram/figure/05-eda-func-form-age-2}
    \label{fig:nomo-funcform-age}}
  \subbottom[Tumour size (centered)]{
    \includegraphics[width=.45\textwidth]{analysis/nomogram/figure/05-eda-func-form-size-2}
    \label{fig:nomo-funcform-size}}
\caption[Prognostic predictor functional forms]{\gls{NSWPCN} prognostic predictor functional forms. Smoothed Cox model martingale residual plots indicate hazard relationships that are approximately linear for centered age (panel a), and piecewise linear for centered tumour size (panel b).  For clarity, plots have been restricted to the residual range $[-1,1]$.}
\label{fig:nomo-funcform}
\end{figure}

\paragraph{Proportional hazards assumption}
A Grambsch-Therneau test \cite{Grambsch1994} on the \gls{CPH} model fit using all expanded terms indicated that patient sex violated the \glspl{PH} assumption ($P = 0.0104$, \fref{fig:nomo-ph-plot-sex}) -- in other words, the two sexes had significantly different baseline hazard shapes.  To account for this effect, all subsequent models were stratified by patient sex, so that the survival of male and female patients was modelled by two different baseline hazard functions.  A Grambsch-Thernau test on the stratified model indicated no further significant violations of the \gls{PH} assumption (global $P = 0.4194$).

\begin{figure}
\centering
  \includegraphics[width=.7\linewidth]{analysis/nomogram/figure/05-eda-ph-check-full-1}
  \caption[Baseline hazard forms differ between patient sexes]{NSWPCN baseline hazard differs between patient sexes.  A natural spline smooth of scaled Schoenfeld residuals for patient sex has a slope obviously differing from zero, suggesting that the baseline hazard forms differ between the two sexes, and that the combined data violates the \acrshort{PH} assumption of Cox regression.  Individual residuals are displayed as points, the natural spline smooth ($\mbox{df}=4$) as a solid line, and approximate $\pm 1\ \mbox{SE}$ bounds as dashed lines.}
\label{fig:nomo-ph-plot-sex}
\end{figure}

\paragraph{Outlier removal}
Strongly influential or outlying samples from the full marginal Cox fit were removed from the \gls{NSWPCN} building set.  I considered this unusual measure to be necessary given known and unresolvable quality issues in the \gls{NSWPCN} cohort data.  For all subsequent work, patients with full marginal Cox model absolute deviance residuals exceeding 2.5, or any absolute DFBETAS score exceeding 0.3, were excluded from the original building set.  This filter removed seven patients, reducing the size of the model building set to 193 patients.	

\paragraph{Variable selection}
Stepwise variable elimination was used to select an AIC-optimal model starting from the full marginal \gls{CPH} model containing all expanded terms and a sex stratum.  The identified optimal \gls{CPH} model used four variables: tumour location (head vs body), tumour size (linear term only), S100A2 status, and S100A4 status, in addition to the sex stratum.  The final \gls{AIC}-selected set of prognostic terms (tumour location, size linear term, S100A2 binary status, S100A4 binary status, and a patient sex stratum) was denoted the reduced term set.

\paragraph{Model CP1}
A final prognostic \acrshort{CPH} regression model was fit to the \gls{NSWPCN} model building data using only the reduced term set; this model was termed CP1.  CP1 did not violate the \gls{PH} assumption by the Grambsch-Therneau test (global $P = 0.794$).  Predictions from model CP1 were broadly concordant with stratified \gls{KM} estimates across all covariate subgroups, indicating no serious lack of fit of the model (\fref{fig:nomo-cp1-gg1-fitplot}).

\paragraph{Model GG1}
Semiparametric Cox \gls{PH} models such as CP1 provide a convenient framework for covariate testing and model diagnostics, but their unspecified baseline hazard term significantly complicates their use as prognostic predictors: patients can only be ranked by relative hazard, and absolute estimates of survival probabilities are unavailable.  Although it is possible to approximate the baseline hazard in the Cox model, a more robust alternative is to use fully parametric models, in which the baseline hazard distribution is explicity specified.  The advantages of parametric models in terms of robustness and interpretability are offset by their more stringent assumptions: if the chosen baseline distribution is unsuited to the particular data to be fit, predictions from parametric models can be very poor.  Given the potential benefits of parametric models for survival prediction, a parametric alternative to model CP1 was developed, and its fit assessed.  This parametric model was termed GG1.

Model GG1, employing a \gls{GG} survival distribution \cite{Cox2007}, was fit to the \gls{NSWPCN} model building data by maximum likelihood.  Guided by the model functional form and baseline hazard stratification indicated by the Cox model diagnostics, the \gls{GG} distribution location parameter $\beta$ was made linearly dependent on all terms in the reduced set, but the shape parameters $\sigma$ and $\lambda$ were modelled as dependent on patient sex only.  The goodness of fit of GG1 was investigated by and graphical assessment of prediction accuracy.  Comparisons between GG1 predictions and \gls{KM} estimates of survival indicated that the GG1 baseline distribution could not accurately model survival in some cohort subsets, particularly the S100A2 negative male group (\fref{fig:nomo-cp1-gg1-fitplot})

\begin{figure}
\centering
  \includegraphics[width=.7\linewidth]{analysis/nomogram/figure/05-final-fit-assessment-4}
  \caption[Model survival predictions agree with stratified \acrshort{KM} estimates]{Model survival predictions agree with stratified \gls{KM} estimates.  \gls{KM} estimates of survival probability for each combination of patient sex and biomarker status are shown as solid blue lines, with 95\% confidence intervals indicated by blue ribbons.  Estimates of survival probability generated by both models CP1 (red) and GG1 (green) broadly followed the form of the \gls{KM} estimator, and lay within its bounds at all times, although the GG1 fit was relatively poor in some strata.  Both model fitting and prediction used the \gls{NSWPCN} model building set, and so these plots illustrate model goodness-of-fit, but cannot indicate possible overfitting.  \gls{KM} traces for the S100A2 positive, S100A4 negative group were omitted, as there were insufficient patients in this group for reliable \gls{KM} estimates to be available.  For all plots, tumour location, and size, were set to cohort median values.}
\label{fig:nomo-cp1-gg1-fitplot}
\end{figure}

\paragraph{Model RSF}
Regression models like CP1 and GG1 are familiar and readily interpretable, but are heavily dependent on the analyst identifying appropriate variables and functional forms.  Ensemble tree models such as random forests \cite{Breiman2001} naturally and automatically model nonlinearity and arbitrary level interactions, and are tolerant of large numbers of irrelevant or collinear variables, albeit at the cost of very poor interpretability, and large data and computational requirements.  Random forests have been adapted to model censored data \cite{Ishwaran2008}, and can provide an alternative prognostic predictor that is distinct in behaviour from CP1 and GG1, and may be able to exploit data structure not leveraged by these more classical models.

To investigate whether tree ensemble models could provide improved performance over classical approaches, a random survival forest model, termed RSF, was fit to the \gls{NSWPCN} model building data.  In contrast to CP1 and GG1, which used the reduced set of terms as covariates, RSF was supplied all preoperatively-assessable variables as candidate predictors.

\paragraph{Model selection}
Predictive performance of the three prognostic models (CP1, GG1, and RSF) was compared on the holdout \gls{NSWPCN} model test set, to select a single high-performing parsimonious model for external validation.  Performance in the interval from seven to 34 months post-diagnosis was of particular interest, as the majority of patients in the \gls{NSWPCN} training set died during this period.  Model GG1 was the overall best-performing model in this $7-34$ month interval, as assessed by both cumulative/dynamic \gls{TDROC} \cite{Blanche2013} (\fref{nomo-tdauc-paths}) and Brier score \cite{Graf1999} (\fref{nomo-brier-paths}).  There was no significant difference between the $7-34$ month \gls{IBS} of competing models, as estimated using 95\% bootstrap confidence intervals, although all models had significantly better \gls{IBS} than a marginal Kaplan-Meier survival estimator (\tref{TODO}).  
%The Brier score provides an overall measure of predictive performance, but it assesses both model discrimination and calibration, whereas it has been argued that high model discrimination is the far more relevant criterion in most clinical applications (TODO cite).  Model discrimination alone can be measured by the \gls{TDROC}-\gls{AUC}, an extension of the familiar \gls{AUC} concept to censored data, that is reflective of the use of prognostic models in clinical practice \cite{Heagerty2005}.
As there was no significant difference in performance between the prognostic models, the simplest model GG1 was selected to form the \gls{PCOP}.

\paragraph{Final \acrshort{PCOP} fit}
A final fit of GG1 to the full \gls{NSWPCN} training data (both model building and validation patients) was made, and is summarised in \tref{tab:nomo-final-fit}.  This fit defined the \gls{PCOP}, which predicts post-resection outcome using a generalized gamma model \cite{Cox2007}, as
\begin{align*}
T \sim GG(\beta   = &\ 6.7446 + 0.3732[\mbox{Sex = Male}] - 0.2150[\mbox{Location = Body}] \\
                    &\ -0.0887\;\mbox{Size} - 0.3729[\mbox{S100A2 = Positive}] \\
                    &\ -0.3843[\mbox{S100A4 = Positive}], \\ 
          \sigma  = &\ 0.7503 - 0.2452[\mbox{Sex = Male}],\\
          \lambda = &\ 0.0288 - 0.7630[\mbox{Sex = Male}])
\end{align*}
where $T$ is an individual's failure time, $GG$ is the generalized gamma distribution, $\mbox{Size}$ is in centimetres, and $[\;]$ is the Iverson bracket.

\begin{figure}
\centering
  \includegraphics[width=.7\linewidth]{analysis/nomogram/figure/05-model-selection-roc-cd-1}
  \caption[Time-dependent \acrshort{AUC} paths for candidate models on holdout data]{Cumulative/dynamic \gls{TDROC} \gls{AUC} paths for candidate models on the holdout \gls{NSWPCN} model test set.}
\label{fig:nomo-tdauc-paths}
\end{figure}

\begin{figure}
\centering
  \includegraphics[width=.7\linewidth]{analysis/nomogram/figure/05-model-selection-bs-paths-1}
  \caption[Brier score paths for candidate models on holdout data]{Brier score paths for candidate models on the holdout \gls{NSWPCN} model test set.  All models outperformed the no-information KM0 trace from approximately 100 days to 600 days post-diagnosis, and no strong differences were apparent between candidate models.}
\label{fig:nomo-brier-paths}
\end{figure}


\begin{table}
\centering
\caption[Final PCOP fit]{Coefficients of a final GG1 fit to the \gls{NSWPCN} training data, which defines the \gls{PCOP}.  Coefficient estimates are for a generalized gamma survival model \cite{Cox2007}.}\label{tab:nomo-final-fit}
\begin{tabular}{llrr}
\toprule
Term                   &                & Estimate       & SE      \\ \midrule
$\beta$ \\
(Intercept)            &                &  $6.7446$      & $0.1489$ \\
Sex                    & = Male         &  $0.3732$      & $0.1508$ \\
Tumour location        & = Body         & $-0.2150$      & $0.1223$ \\
Size of longest axis   &  (cm)          & $-0.0887$      & $0.0302$ \\
S100A2                 & = Positive     & $-0.3729$      & $0.1235$ \\
S100A4                 & = Positive     & $-0.3843$      & $0.1045$ \\[10]
$\sigma$ \\
(Intercept)            &                &   $0.7503$     & $0.0493$ \\
Sex                    & = Male         &  $-0.2452$     & $0.1066$ \\[10]
$\lambda$ \\
(Intercept)            &                &  $0.0288$      & $0.2719$ \\
Sex                    & = Male         &  $0.7630$      & $0.3533$ \\[5] \bottomrule
\end{tabular}
\end{table}


\subsection{External validation}


\subsection{Web tool}
\hyperref[http://54.66.150.159:3277/]{http://54.66.150.159:3277/}

TODO: In app, show +/- margin curves, to guide surgeons as to benefit from aggressive surgery?  Definite counfounding potential

\subsection{Cohort characteristics}
\label{subsec:nomo-results-cohort}

\subsection{Development of a preoperative prognostic model}

\subsection{Validation of the prognostic model}
Intro here on disc \& calib.  Also describe cohorts briefly.
\subsubsection{Discrimination}
\subsubsection{Calibration}

\subsection{PCOP web application}

\section{Discussion}

\section{Methods}
\subsection{Cohort recruitment and ethics}
\label{subsec:nomo-methods-cohort}

\subsection{Model building and selection}
All statistical modelling was performed within the \texttt{R} environment.  \gls{CPH} and \gls{DM} models were fit and analysed using the base package \texttt{survival}, and Cox model stepwise variable elimination was performed using the function \texttt{stepAIC} from package \texttt{MASS}.  Generalised gamma survival models were fit using the implementation in package \texttt{flexsurv}\footnote{Parameter symbols differ between the \texttt{flexsurv} package, and this chapter and \cite{Cox2007}.  In this chapter and \cite{Cox2007}, the generalized gamma location parameter is denoted $\beta$, and shape parameters are $\sigma$, and $\lambda$.  In \texttt{flexsurv}, these parameters are denoted $\mu$, $\sigma$, and $Q$, respectively.}, and package \texttt{randomForestSRC} supplied random survival forest functions.  The random survival forest model was trained with parameters \texttt{splitrule = "logrankscore"}, \texttt{nsplit = 2}, and \texttt{ntree = 1000}, with all other parameters set to defaults.

Both the cumulative/dynamic \gls{TDROC}, and the \gls{IBS}, were used to compare model prognostic performance.  \glspl{TDROC} were estimated using R package \texttt{timeROC}, and Brier score paths and \glspl{IBS} were calculated with custom code, following \cite{Graf1999}.

\subsection{External validation}

\subsection{PCOP web application}
The \texttt{R} \texttt{shiny} infrastructure was used to create a simple web application to predict patient outcome using the final \gls{PCOP} model.

\end{document}
